\section{Nesting capabilities, 1-WAYS and 2-WAYS using the AGRIF procedure}

\subsection{Introduction}
\label{sec:introduction}

To address the challenge of bridging the gap between near-shore and offshore
dynamics, a nesting capability has been added to ROMS and tested for the California
Upwelling System \citep{Pen04}.  The method chosen for embedded griding takes
advantage of the AGRIF (Adaptive Grid Refinement in Fortran) package
\citep{Bla99,Deb00, Deb03a,Deb03b,Deb08}.  AGRIF is a Fortran 95 package for the
inclusion of adaptive mesh refinement features within a finite difference numerical
model. One of the major advantages of AGRIF in static-grid embedding is the ability
to manage an arbitrary number of fixed grids and an arbitrary number of embedding
levels.

\begin{figure}[htbp]
\centerline{\psfig{figure=nesting_fig1.eps,width=15cm}}
\caption{Temporal coupling between a parent and a child grid for a refinement factor
  of 3.  The coupling is done at the baroclinic time step.}
\label{fig:temp_coupling}
\end{figure}

A recursive integration procedure manages the time evolution for the child grids
during the time step of the parent grids (Figure \ref{fig:temp_coupling}). In order
to preserve the CFL criterion, for a typical coefficient of refinement (say, a factor
of 3 for a 5 km resolution grid embedded in a 15 km grid), for each parent time step
the child must be advanced using a time step divided by the coefficient of refinement
as many time as necessary to reach the time of the parent (Figure
(\ref{fig:temp_coupling})).  For simple 2-level embedding, the
procedure is as follows:\\
\begin{enumerate}
\item Advance the parent grid by one parent time step.
\item Interpolate the relevant parent variables in space and time to get the boundary
  conditions for the child grid.
\item Advance the child grid by as much child time steps as necessary
to reach the new parent model time.
\item Update point by point the parent model by averaging the more accurate values of
  the child model (in the case of 2-way embedding).
\end{enumerate}
The recursive approach used in AGRIF allows the specification of any number of
embedding level. Other cpp keys are related to AGRIF, they are in
  set\_global\_definitions.h and set\_obc\_definitions.h files. These ones are the
  default conditions, are located in the ROMS\_AGRIF code sources and should not be
  edit by standard user.

\subsection{2-WAYS nesting:  feed-back from child to parent grid}
\label{sec:2-ways-nesting}
\textbf{To be continued ...}


% \subsection{Specification in the ROMS\_AGRIF $2.0$ code}\label{sec:spec_romsagrif2.0}
% Here are the main cpp-keys related to AGRIF nesting procedure.
% \begin{itemize}
% \item AGRIF\_OBC\_EAST : Open eastern boundary for the child grids.
% \item AGRIF\_OBC\_WEST : Open western boundary for the child grids.
% \item AGRIF\_OBC\_SOUTH : Open southern boundary for the child grids.
% \item AGRIF\_OBC\_NORTH : Open northern boundary for the child grids.
% \item AGRIF\_FLUX\_BC : Apply parent/child barotropic boundary conditions as fluxes.
% \item AGRIF\_OBC\_M2FLATHER :  Activate Flather open boundary conditions for ubar and vbar
% for the child model .
% \item AGRIF\_OBC\_M2ORLANSKI : Activate 2D radiation open boundary conditions for ubar and vbar
% for the child model.
% \item AGRIF\_OBC\_M2SPECIFIED : Activate specified open boundary conditions for ubar and vbar
% for the child model.
% \item AGRIF\_OBC\_M2CHARACT : Activate open boundary conditions based on
%   characteristic methods for ubar and vbar for the child model.
% \item AGRIF\_OBC\_M3ORLANSKI : Activate 2D radiation open boundary conditions for u and v
% for the child model.
% \item AGRIF\_OBC\_M3SPECIFIED : Activate specified open boundary conditions for u and v
% for the child model.
% \item AGRIF\_OBC\_M3CHARACT : Activate open boundary conditions based on characteristic methods 
% for u and v for the child model.
% \item AGRIF\_OBC\_TORLANSKI :  Activate 2D radiation open boundary conditions for tracers
% for the child model .
% \item AGRIF\_OBC\_TUPWIND : Activate upwind open boundary conditions for tracers
% for the child model.
% \item AGRIF\_OBC\_TSPECIFIED  : Activate specified open boundary conditions for tracers
% for the child model.
% \end{itemize}

% \begin{center}
%   \bf The default definitions should be sufficient for most of the applications.
%\end{center}
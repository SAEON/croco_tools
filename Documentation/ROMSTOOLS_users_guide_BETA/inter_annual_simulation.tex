
ROMSTOOLS can help to realize inter-annual simulations. In this context, 
we rely on Ocean Global Circulations Models (OGCM) for the lateral 
boundary conditions and a global atmospheric reanalysis for the surface 
forcing (NCEP). To limit the volume of data which needs to be transfered 
over the Internet, we use Opendap to extract only the necessary subgrids. 

\subsection{Getting the surface forcing data from NCEP}
The Matlab script make\_NCEP.m is used to obtain the surface forcing data.  It
downloads the necessary NCEP surface forcing data (Sea Surface Temperature, Wind
stress ...) over the Internet, and interpolates them on the model grid. Since
make\_NCEP.m works with the bulk parameterization (i.e. the BULK\_FLUX and BULK\_EP
cpp keys should be defined in cppdefs.h), a surface forcing NetCDF file and a bulk
NetCDF file are generated for each month of your simulation in the directory
$\sim$/Roms\_tools/Run/ROMSFILES/ .
The part of the file romstools\_param.m that you should change is:\\\\
\%\\
\%\%\%\%\%\%\%\%\%\%\%\%\%\%\%\%\%\%\%\%\%\%\%\%\%\%\%\%\%\%\%\%\%\%\\
\%\\
\% 7 Parameters for Interannual forcing (SODA, ECCO, NCEP, ...)\\
\%\\
\%\%\%\%\%\%\%\%\%\%\%\%\%\%\%\%\%\%\%\%\%\%\%\%\%\%\%\%\%\%\%\%\%\%\\
Download\_data = 1;             \% Get data from the OPENDAP sites   \\
level = 0;                               \% AGRIF level; 0=parent grid  \\
\%	 \\
NCEP\_version  = 2;                            \% NCEP version:  \\
\% 1: NCEP/NCAR Reanalysis, 1/1/1948 - present  \\
\% 2: NCEP-DOE Reanalysis, 1/1/1979 - present  \\
\%					       \\

\noindent \% Path and option for using global datasets download from ftp  \\
\%  \\
Get\_My\_Data = 0;   \\
\%  \\

\noindent if NCEP\_version  == 1;  \\
My\_NCEP\_dir  = [DATADIR,'NCEP\_REA1/'];  \\
elseif NCEP\_version  == 2;  \\
My\_NCEP\_dir  = [DATADIR,'NCEP\_REA2/'];  \\
end  \\

\noindent My\_QSCAT\_dir = [DATADIR,'QSCAT/'];  \\
My\_SODA\_dir  = [DATADIR,'SODA/'];  \\
My\_ECCO\_dir  = [DATADIR,'ECCO/'];  \\
  \\

\noindent \%======================================\\
\%Options for make\_NCEP and make\_QSCAT\_daily  \\
\%  \\
NCEP\_dir= [FORC\_DATA\_DIR,'NCEP\_',ROMS\_config,'/'];  \% NCEP data directory \\
makefrc     = 1;                      \% 1: Create forcing files \\
makeblk     = 1;                       \% 1: Create bulk files \\
QSCAT\_blk  = 1; \% Correct NCEP frc/bulk file with the u,v,wspd fields from QSCAT daily
data. Download u, v, wspd in the QSCAT frc file  \\ 

\noindent add\_tides = 0; \% 1: Add the tides (To be done...) \\


\noindent \textit{\%Overlap parameters :}  \\
itolap\_qscat=11;  \%11 days if 1d time reso. QSCAT (should be <28 ) \\
itolap\_ncep=40;   \%10 days if 6h time res.  NCEP  (should be <4* 28 =112 \\
\% ... \\

Variables description :
\begin{itemize}
\item FORC\_DATA\_DIR : Directory where the different files downloaded over 
the Internet are stored.
\item Download\_data : Get data from the OPENDAP sites. Should be 1.
\item level : AGRIF level. The parent grid = 0 and the child grid = 1.
\item NCEP\_dir= [FORC\_DATA\_DIR,'NCEP\_',ROMS\_config,'/'] : NCEP data directory. 
This is where NCEP data downloaded over the Internet are stored.
\item makefrc : Switch to define if the forcing file is generated. Should be 1.
\item makeblk : Switch to define if the bulk file is generated. Should be 1.
\item add\_tides : Switch to define if the tidal forcing is added. 
\item NCEP\_version : version of the NCEP reanalysis. 1: NCEP/NCAR Reanalysis, 1/1/1948 - present.
2: NCEP-DOE Reanalysis, 1/1/1979 - 12/31/2001.
\item Get\_My\_Data = 1
\item My\_NCEP\_dir = Path to local global NCEP datasets
\item QSCAT\_blk = Flag to use the QiuikSCAT wind in the NCEP bulk files
\item itolap\_qscat = Overlap parameters for the monthly roms forcing files using
  QuikSCAT daily wind stress.  \\
The overlap parameter is the number of "recovering"
  time steps between 2 consecutives months
\item itolap\_ncep = Overlap parameters for the monthly roms forcing (and/or bulk files) using NCEP1 or
  NCEP2 wind stress( and/or heat fluxes)  monthly file
\end{itemize}

Save romstools\_param.m and run make\_NCEP in the Matlab session.

\subsubsection{Using OpenDAP : Get\_My\_Data = 0}

You should obtain:
\\\\
$>>$ make\_NCEP\\
Read in the grid ROMS\_FILES/roms\_grd.nc \\
\\ 
$===========================================$ \\
BEGIN DOWNLOAD STEP   \\
$===========================================$ \\
$===========================================$ \\
Download NCEP data with OPENDAP or my FTP data \\
$===========================================$ \\
$===========================================$ \\
OPENDAP Procedure \\
$===========================================$ \\
 
\noindent Get NCEP data from 2000 to 2000 \\
From http://nomad1.ncep.noaa.gov:9090/dods/reanalyses/reanalysis-2/ \\

\noindent Minimum Longitude: 8 \\
Maximum Longitude: 22 \\
Minimum Latitude: -38 \\
Maximum Latitude: -25.8968
\noindent Making output data directory \\
......./Run/DATA/NCEP\_Benguela\_LR/ \\
$=========================================$ \\
VNAME IS landsfc \\
$=========================================$ \\
$------$ \\
Get time units and time:  Get\_My\_Data is OFF  \\
$------$ \\
Reading: http://nomad1.ncep.noaa.gov:9090/dods/reanalyses/reanalysis-2/6hr/flx/flx \\
  Constraint: time \\
Server version: dods/3.2\\
...


\subsubsection{Using FTP global dataset : Get\_My\_Data $=$ 1}
$>>$ make\_NCEP \\
Read in the grid ROMS\_FILES/roms\_grd.nc \\
\\ 
$============================================$ \\
Download NCEP data with OPENDAP or my FTP data \\
$============================================$ \\
$============================================$ \\
Direct FTP Procedure \\
$============================================$ \\
Use my own ncep data NCEP$2$ \\ 
 
\noindent Get NCEP data from $2000$ to $2000$ \\
From path/NCEP\_REA2/  \\
 
\noindent Minimum Longitude: $8$ \\
Maximum Longitude: $22$ \\
Minimum Latitude: $-38$ \\
Maximum Latitude: $-25.8968$ \\

\noindent $============================================$ \\
Get\_My\_Data = 1 \\
Read subgrid from file/data1/gcambon/NCEP\_REA2/land.sfc.gauss.nc \\
$============================================$ \\
\noindent Get the Land Mask tindex = 1 \\
\noindent In case of Get\_My\_Data ON \\

\noindent Get the Land Mask by using extract\_NCEP\_Mask\_Mydata \\
\noindent Execute extract\_NCEP\_Mask\_Mydata \\
 
 
\noindent Get land for year 2000 - month 1 \\

\noindent Create \textit{path}/Run/DATA/NCEP\_Benguela\_LR/land\_Y2000M1.nc \\
$============================================$ \\
VNAME IS air \\
$============================================$ \\
\\
$============================================$ \\
Processing year: 2000 \\
$============================================$ \\





\subsection{Getting the surface windstress data from QuickSCAT}

\subsubsection{QuikSCAT daily data from Ifremer OpenDap server}
\label{sec:quikscat-daily-data}


Similarly, The Matlab script make\_QuickSCAT\_daily.m is used to obtain the daily
surface stress forcing provided by the OpenDAP server at Ifremer, France: \\
http://www.ifremer.fr/dodsG/CERSAT/quikscat\_daily. \\
A surface forcing NetCDF file NetCDF file is generated for each month of your
simulation in the directory $\sim$/Roms\_tools/Run/ROMSFILES/. \\

\noindent You shoud edit this part of the file romstools\_param.m: \\


\noindent \%\%\%\%\%\%\%\%\%\%\%\%\%\%\%\%\%\%\%\%\%\%\%\%\%\%\%\%\%\%\%\%\%\%\%\%\%\%\%\\
\%\\
\% 7 Parameters for Interannual forcing (SODA, ECCO, NCEP, ...)\\
\%\\
\%\%\%\%\%\%\%\%\%\%\%\%\%\%\%\%\%\%\%\%\%\%\%\%\%\%\%\%\%\%\%\%\%\%\%\%\%\%\%\\
\%\\
\% Path to Forcing data\\
.....\\
\%  Options for make\_QSCAT\_daily and make\_QSCAT\_clim\\
\%\\
QSCAT\_dir        = [FORC\_DATA\_DIR,'QSCAT\_',ROMS\_config,'/'];\% QSCAT data directory.\\
QSCAT\_frc\_prefix = [frc\_prefix,'\_QSCAT\_'];  \% generic forcing file name for interannual roms simulations with QuickSCAT.\\
QSCAT\_clim\_file = [DATADIR,'QuikSCAT\_clim/roms\_QSCAT\_month\_clim\_2000\_2007.nc'];   \% QuikSCAT climatology file for make\_QSCAT\_clim.\\
\%
\%\%\%\%\%\%\%\%\%\%\%\%\%\%\%\%\%\%\%\%\%\%\%\%\%\%\%\%\%\%\%\%\%\%\%\%\%\%\\


\noindent In a Matlab session, run make\_QSCAT\_daily  \\
$>>$ \\
$>>$ make\_QSCAT\_daily \\

\noindent if you download data over the internet using OpenDAP, you should obtain that during
the dowload step : 

\noindent $>>$ \\ 
...\\ 
Reading: http://www.ifremer.fr/dodsG/CERSAT/quikscat\_daily\\
  Constraint: mwst[167:167][78:113][370:409]\\
Server version: apache-coyote/1.1\\
    Processing day: 1\\
\\
Reading: http://www.ifremer.fr/dodsG/CERSAT/quikscat\_daily\\
  Constraint: zwst[168:168][78:113][370:409]\\
Server version: apache-coyote/1.1\\
\\
Reading: http://www.ifremer.fr/dodsG/CERSAT/quikscat\_daily\\
  Constraint: mwst[168:168][78:113][370:409]\\
Server version: apache-coyote/1.1\\
    Processing day: 2\\
....\\
$>>$\\ 
\subsubsection{QuikSCAT monthly climatology data}
\label{sec:quikscat-daily-data}
\noindent If you want to use the QSCAT climatology, computed over $2000$-$2007$, based over
these previous QSCAT data, in a Matlab session, run make\_QSCAT\_clim.

\noindent $>>$ \\
$>>$ make\_QSCAT\_clim \\
$>>$ ...\\

\subsection{Getting the lateral boundary conditions}

Initial conditions and lateral boundary conditions and  can be 
obtained from several ocean global circulation models (OGCM) 
such as SODA \citep{Car05} or ECCO \citep{Sta99}. The SODA 
reanalysis is available from 1958 to 2001 and ECCO is available 
from 1993 until now. The Matlab script make\_OGCM.m is used to 
download data over the Internet, and to perform the interpolations 
on the model grid. 
A lateral boundary conditions NetCDF file is generated for each month 
of your simulation in the directory $\sim$/Roms\_tools/Run/ROMSFILES/ . 
\\
\\
The part of the file romstools\_param.m that you should change is:
\\
\\
\%\%\%\%\%\%\%\%\%\%\%\%\%\%\%\%\%\%\%\\
\%\\
\% Options for make\_OGCM \\
\%\\
\%\%\%\%\%\%\%\%\%\%\%\%\%\%\%\%\%\%\%\\
OGCM        = 'SODA';                                \% Select the OGCM:
SODA(1958-2001), ECCO(1993-2005), ...\\
OGCM\_dir    = [FORC\_DATA\_DIR,OGCM,'\_',ROMS\_config,'/'];  \\
bry\_prefix  = [ROMS\_files\_dir,'roms\_bry\_',OGCM,'\_']; \\
clm\_prefix  = [ROMS\_files\_dir,'roms\_clm\_',OGCM,'\_']; \\
ini\_prefix  = [ROMS\_files\_dir,'roms\_ini\_',OGCM,'\_']; \\
OGCM\_prefix = [OGCM,'\_'];                            \\
rmdepth     = 2;                                    \\ \\
\noindent \%Overlap parameters : before (\_a) and after (\_p) the months.\\ 
itolap\_a=2;~~~\%Overlap parameter at the begining of the months. \\ 
itolap\_p=2;~~~\%Overlap parameter at the end of the months. \\ 
\%                        \\
\\
Variables description :
\begin{itemize}
\item OGCM = 'SODA' : Name of the OGCM employed (SODA or ECCO).
\item OGCM\_dir    = [FORC\_DATA\_DIR,OGCM,'\_',ROMS\_config,'/']  : 
OGCM data directory.
\item bry\_prefix  = [ROMS\_files\_dir,'roms\_bry\_',OGCM,'\_'] : 
Left part of the boundary file name.
\item clm\_prefix  = [ROMS\_files\_dir,'roms\_clm\_',OGCM,'\_'] : 
Left part of the climatology file name.
\item ini\_prefix  = [ROMS\_files\_dir,'roms\_ini\_',OGCM,'\_'] : 
Left part of the initial file name.
\item OGCM\_prefix = [OGCM,'\_'] : 
Left part of the OGCM file name. This is where OGCM data are
stored. 
\item rmdepth = 2 : Number of bottom levels to remove.
This is useful when there is no valid data at this level.
For example, if the depth in the ROMS domain is shallower 
than the OGCM depth.
\item itolap\_a : Overlap parameter at the \underline{begining} of a month with the
  next month. 
\item itolap\_p : Overlap parameter at the \underline{end} of a months with the
  previous months.\\
Commonly, these two parameters are equal. 
\end{itemize}
Save romstools\_param.m and run make\_OGCM in the Matlab session.
You should obtain:
\\\\
$>>$ make\_OGCM\\
Add the paths of the different toolboxes\\
Arch : x86\_64 - Matlab version : 2006a\\
Use of mexnc and loaddap in 64 bits.\\
Download data...\\
\\
Get data from Y2000M1 to Y2000M3\\
Minimum Longitude: 12.3\\
Maximum Longitude: 20.3\\
Minimum Latitude: -35.5\\
Maximum Latitude: -26.3815\\
\\
Making output data directory ../Run/DATA/SODA\_Benguela/\\
Process the dataset: http://iridl.ldeo.columbia.edu./SOURCES/.CARTON-GIESE/.SODA/.v1p4p3\\
Processing year: 2000\\
  Processing month: 1\\
    Download SODA for 2000 - 1\\
    ...SSH\\
    ...U\\
...

\subsection{Running the model for interannual runs}

Compile the model with jobcomp (and with the 
cpp keys BULK\_FLUX and BULK\_EP defined) and edit 
the input parameter file 
$\sim$/Roms\_tools/Run/roms\_inter.in as for the
climatology experiments. As for the long simulations, a csh script
(run\_roms\_inter.csh) manages the handling of input and output files.
It also changes the number of time steps so each month has the correct
length. This script takes care of leap years. For example Y1996M2 
(February 1996) is 29 days long.

Part to edit in run\_roms\_inter.csh:\\
\\
\#\\
set MODEL=roms\\
set SCRATCHDIR=`pwd`/SCRATCH\\
set INPUTDIR=`pwd`\\
set MSSDIR=`pwd`/ROMS\_FILES\\
set MSSOUT=`pwd`/ROMS\_FILES\\
set CODFILE=roms\\
set AGRIF\_FILE=AGRIF\_FixedGrids.in\\
\#\\
set BULK\_FILES=1\\
set FORCING\_FILES=1\\
set CLIMATOLOGY\_FILES=0\\
set BOUNDARY\_FILES=1\\
\#\\
\# Atmospheric surface forcing dataset (NCEP, GFS,...)\\
\#\\
set ATMOS=NCEP\\
\#\\
\# Oceanic boundary and initial dataset (SODA, ECCO,...)\\
\#\\
set OGCM=SODA\\
\#\\
\# Model time step [seconds]\\
\#\\
set DT=5400\\
\#\\
\# number total of grid levels (1: No child grid)\\
\#\\
set NLEVEL=1\\
\#\\
set NY\_START=2000\\
set NY\_END=2000\\
set NM\_START=1\\
set NM\_END=3\\
\#\\
\#  Restart file - RSTFLAG=0 --$>$ No Restart\\
\#		  RSTFLAG=1 --$>$ Restart\\
\#\\
set RSTFLAG=0\\
\#\\
\#  Time Schedule  -  TIME\_SCHED=0 --$>$ yearly files\\
\#                    TIME\_SCHED=1 --$>$ monthly files\\
\#\\
set TIME\_SCHED=1\\
\#\\
\#\#\#\#\#\#\#\#\#\#\#\#\#\#\#\#\#\#\#\#\#\#\#\#\#\#\#\#\#\#\#\#\#\#\#\#\#\#\#\#\#\#\#\#\#\#\#\#\#\#\#\#\#\#\#\#
\\
\\
Variables definitions:
\begin{itemize}
\item MODEL=roms : Name used for the input files. For example roms\_grd.nc.
\item SCRATCHDIR=`pwd`/SCRATCH : Scratch directory where the model is run.
\item INPUTDIR=`pwd` : Input directory where the roms\_inter.in input file
is located.
\item MSSDIR=`pwd`/ROMS\_FILES : Directory where the roms input NetCDF files
(roms\_grd.nc, roms\_frc.nc, ...) are stored.
\item MSSOUT=`pwd`/ROMS\_FILES : Directory where the roms output NetCDF files
(roms\_his.nc, roms\_avg.nc, ...) are stored.
\item CODFILE=roms : ROMS executable.
\item AGRIF\_FILE=AGRIF\_FixedGrids.in : AGRIF input file which defines the 
position of child grids when using embedding.
\item BULK\_FILES=1 : 1 if using bulk NetCDF files (should be 1 for NCEP).
\item FORCING\_FILES=1 : 1 if using forcing NetCDF files (should be 1 for NCEP).
\item CLIMATOLOGY\_FILES=0 : 1 if using XXX\_clm.nc files. Using a climatology
file for each month can take a lot of disc space. It is less costly to use
 boundary files (XXX\_bry.nc). 
\item BOUNDARY\_FILES=1 : 1 if using XXX\_bry.nc files. 
\item ATMOS=NCEP : name of the atmospheric reanalysis. For the moment it is only
NCEP.
\item OGCM=SODA : name of the OGCM for the boundary conditions. SODA or ECCO.
\item DT=5400 : Model time step in seconds.
\item NDAYS = 30 : Number of days in 1 month.
\item NLEVEL=1 : Total number of model grids (no embedding: NLEVEL=1).
\item NY\_START=2000 : Starting year.
\item NY\_END=2000 : Ending Year.
\item NM\_START=1 : Starting month.
\item NM\_END=3 : Ending month.
\item RSTFLAG=0 : 1 if restarting a simulation
\item TIME\_SCHED=1 : (obsolete) 0 if using yearly files, 1 if using monthly 
files. Since make\_NCEP and make\_OGCM are creating only monthly
files, it should be always 1.
\end{itemize}


As for ROMS long climatology experiments, inter-annual experiments can be run
in batch mode:\\
$>$: nohup ./run\_roms\_inter.csh $>$ exp1.out \&\\\\

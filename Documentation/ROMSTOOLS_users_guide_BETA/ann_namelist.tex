\section{Namelist description : \textit{roms.in}}\label{namelistdesc}

\subsection{Exemple of South Benguela Test Case}
\textbf{title}: \\
South Benguela TEST MODEL \\
\textbf{time\_stepping}: NTIMES   dt[sec]  NDTFAST  NINFO \\
$720$      $3600$     $60$      $1$ \\
\textbf{S-coord}: THETA\_S,   THETA\_B,    Hc (m) \\
$7.0d0$      $0.0d0$      $5.0d0$ \\
\textbf{grid}:  filename \\
roms\_grd.nc \\
\textbf{forcing}: filename \\
roms\_frc.nc \\
\textbf{bulk\_forcing}: filename \\
roms\_bulk.nc \\
\textbf{climatology}: filename \\
roms\_clm.nc\\
\textbf{boundary}: filename \\
roms\_bry.nc \\
\textbf{initial}: NRREC  filename \\
$1$  \\
roms\_ini.nc \\
\textbf{restart}:          NRST, NRPFRST / filename \\
$720$    $-1$ \\
roms\_rst.nc \\
\textbf{history}: LDEFHIS, NWRT, NRPFHIS / filename  \\
$T$      $72$     $0$ \\
roms\_his.nc \\
\textbf{averages}: NTSAVG, NAVG, NRPFAVG / filename \\
$1$      $72$     $0$ \\
roms\_avg.nc \\
\textbf{primary\_history\_fields}: zeta UBAR VBAR  U  V   wrtT(1:NT) \\
T    T   T   T  T    $30$*T \\
\textbf{auxiliary\_history\_fields}:   rho Omega  W  Akv  Akt  Aks  HBL HBBL Bostr Wstr UWstr VWstr rsw rlw lat sen HEL \\
F   F     T   F    T    F    T   T    T     T    T    T $10$*F \\
\textbf{primary\_averages}: zeta UBAR VBAR  U  V   wrtT(1:NT) \\
T    T    T    T  T   $30$*T      \\
\textbf{auxiliary\_averages}: rho Omega  W  Akv  Akt  Aks  HBL HBBL Bostr Wstr Wstr UWst  rsw rlw lat sen HEL \\
F   T     T   F    T    F    T   T    T     T   T    T     $10$*F\\
\textbf{rho0}: \\
$1025.d0$ \\
\textbf{lateral\_visc}:   VISC2,    VISC4    [$m^{2}$/sec for all] \\
$0.$       $0.$\\
\textbf{tracer\_diff2}: TNU2(1:NT)           [$m^2$/sec for all] \\
$30*0.d0$ \\
\textbf{tracer\_diff4}: TNU4(1:NT)           [$m^4$/sec for all] \\
$30*0.d11$ \\
\textbf{vertical\_mixing}: Akv\_bak, Akt\_bak [$m^2$/sec] \\
$0.$d$0$    $30*0.$d$0$ \\
\textbf{bottom\_drag}:     RDRG [m/s],  RDRG2,  Zob [m],  Cdb\_min, Cdb\_max \\
$3.0$d$-04$ $0.$d$-3$ $0$.d$-3$ $1$.d$-4$ $1$.d$-1$
\\
\textbf{gamma2}: \\
$1.$d$0$ \\
\textbf{sponge}:          X\_SPONGE [m],    V\_SPONGE [$m^2$ / sec] \\
$150.e3$           $100.$ \\
\textbf{nudg\_cof}:    TauT\_in, TauT\_out, TauM\_in, TauM\_out  [days for all]\\
$1.$       $360.$      $3.$      $360.$ \\
\textbf{diagnostics}:   ldefdia   nwrtdia    nrpfdia /filename \\
T        $72$         $0$ \\
roms\_dia.nc \\
\textbf{diagnostics\_history\_fields \footnote{For each flag, you have to define T or
    F fior the tracer corresponding. In case of physical experiments, we have two
    tracers, T and S, so we have to define 2 logical value for each terms. In case of
    PISCES biogeochemical experiment, we have T, S and $24$ more tracers. So we have
    to define $26$ logical value for each term. For example, in this case replace T T
    by T T 24*T if you want all the PISCES biogeochemical tracer diagnostics}}:
TXadv(1:NT !!EXACT!!) TYadv TVadv THmix TVmix ... Tforc Trate TXadvml TYadvml TVadvml
THmixml TVmixml Tforcml Tentml
Trateml \\
\noindent  T T $~~$      T T $~~$    T T  $~~$    T T  $~~$    T T $~~$      T T $~~$       T T $~~$
 T T  $~~$    T T $~~$      T T  $~~$       T T  $~~$      T T  $~~$  T T $~~$  T T $~~$
 $~~$ T T $~~$   T T $~~$ \\
\textbf{diag\_avg}: ldefdia\_avg  ntsdia\_avg  nwrtdia\_avg  nprfdia\_avg /filename \\
               T          $1 $          $72$            0 \\
                                 roms\_dia\_avg.nc \\
\textbf{diagnostics\_average\_fields}:   TXadv(1:NT !!EXACT!!) TYadv  TVadv  THmix 
TVmix ... Tforc  Trate  TXadvml  TYadvml   TVadvml  THmixml TVmixml Tforcml Tentml
Trateml \\
\noindent  T T $~~$      T T $~~$    T T  $~~$    T T  $~~$    T T $~~$      T T $~~$       T T $~~$
 T T  $~~$    T T $~~$      T T  $~~$       T T  $~~$      T T  $~~$  T T $~~$  T T $~~$
 $~~$ T T $~~$   T T $~~$ \\
\textbf{diagnosticsM}:   ldefdiaM   nwrtdiaM    nrpfdiaM /filename \\
                   T          $72$          $0$ \\
                                 roms\_diaM.nc \\
\textbf{diagnosticsM\_history\_fields \footnote{In the case of momentum equation term,
  we have only $2$ logical flags to define by diagnostic term, one for u and one for v}}:   MXadv MYadv MVadv MCor MPrsgrd MHmix MVmix MRate \\
\noindent  T T$~~$   T T$~~$    T T$~~$ T T$~~$ T T$~~$ T T$~~$ T T$~~$ T T$~~$  \\
\textbf{diagM\_avg}: ldefdiaM\_avg  ntsdiaM\_avg  nwrtdiaM\_avg  nprfdiaM\_avg /filename \\
               T          $1$           $72$            $0$ \\
                                 roms\_diaM\_avg.nc \\
\textbf{diagnosticsM\_average\_fields}:   MXadv MYadv MVadv MCor MPrsgrd MHmix MVmix
MRate \\
\noindent  T T$~~$   T T$~~$    T T$~~$ T T$~~$ T T$~~$ T T$~~$ T T$~~$ T T$~~$  \\
\textbf{diagnostics\_bio}:   ldefdiabio   nwrtdiabio    nrpfdiabio /filename \\
                        T          $72$             $0$ \\
                                 roms\_diabio.nc \\
\textbf{diagbio\_avg}: ldefdiabio\_avg  ntsdiabio\_avg  nwrtdiabio\_avg  nprfdiabio\_avg
/filename \\
                  T              $1$             $72$              $0$ \\
                                 roms\_diabio\_avg.nc \\
\textbf{biology}:   forcing file \\
           roms\_frcbio.nc \\
\textbf{sediments}: input file \\
           sediment.in \\
\textbf{sediment\_history\_fields}: bed\_thick bed\_poros be3d\_fra(sand,silt) \\
                            T         F        T  T \\
\textbf{bbl\_history\_fields}: Abed Hripple Lripple Zbnot Zbapp Bostrw \\
                     T      F       F      T     F     T \\
\textbf{floats}: LDEFFLT, NFLT, NRPFFLT / inpname, hisname \\
           T       $6$      $0$ \\
                                   floats.in \\
                                   floats.nc \\
\textbf{float\_fields}:  Grdvar Temp Salt Rho Vel \\
                 F     F    F    F   F \\
\textbf{stations}: LDEFSTA, NSTA, NRPFSTA / inpname, hisname \\
             T      $400$      $0$ \\
                                    stations.in \\
                                    stations.nc \\
\textbf{station\_fields}:  Grdvar Temp Salt Rho Vel \\
                   T     T    T    T   T \\
\textbf{psource}:   Nsrc  Isrc  Jsrc  Dsrc  Qbar [m3/s]    Lsrc        Tsrc \\
            $2$ \\
            $3$    $54$     $1$    $200. $   T T      $5. 0.$ \\
            $3$    $40$     0    $200.$      T T      $5. 0.$ \\
 

\subsection{Description}
%%%%%%%%%%%%%%%%%%%%%%%%%%%%%%%%%%%%%%%%%%%%%%%%%%%%%%%%%%%%%%%%%%%%%%%%%%%%%%%
%%%%%%%%%%%%%%%%%%%%%%%%%%%%%%%%%%%%%%%%%%%%%%%%%%%%%%%%%%%%%%%%%%%%%%%%%%%%%%%
\begin{longtable}{|p{0.25\linewidth}|p{0.75\linewidth}|}
 \hline
 \textsc{Keywords}  &  \textsc{Descriptions} \\
 \endfirsthead
\hline
 \textsc{Keywords}  &  \textsc{Descriptions} \\
   \hline
   \endhead
   \hline
   \multicolumn{2}{|p{0.6666\linewidth}|}{\textit{Next page}$\rightarrow$} \\
   \hline
   \endfoot
    \endlastfoot 
\hline
\large{\textbf{title}} &  \\
&   \\ 
\large{\textbf{time\_stepping}} &     \\ 
&   NTIMES : Number of time step for the run. \\ 
&   dt  : Baroclinic time step for the run [in s] \\ 
&   NDTFAST : Number of bariotropic time step in one baroclinic time step.  \\
&   NINFO : frequency of output in time steps. \\

\large{\textbf{S-coord}} &    \\ 
&  THETA\_S: $s$-coordinate surface control parameter,
$0$ $<$ \textbf{theta\_s} $<$ $20$ \\ 
&  THETA\_B: $s$-coordinate bottom control parameter,
$0$ $<$ \textbf{theta\_b} $<$ $1$ \\ 
&  Hc(m): Width of the surface or bottom topography layer in which higher vertical
resolution is required during stretching. \\ 

\large{\textbf{grid}} &    \\
&/filename:  Name of the grid file. \\ 
&  \\ 

\large{\textbf{forcing}} &     \\
&/filename: Name of the surface forcing file : wind stress, atmospheric fluxes (E-P,
net heat fluxes) and nudging coefficients towards dQ/dSST.\\
&  \\

\large{\textbf{bulk\_forcing}}    &  \\ 
&/filename:  Name of the bulk forcing file for atmospheric forcings.\\ 


\large{\textbf{climatology} }   &  \\
&/filename: Name of the open boundaries conditions (t, s, $\bar u$, $\bar v$, $u$,
$v$). These files are 3d in space, covering the whole domain. \\ 
&  \\ 


\large{\textbf{boundary}} &  \\
&/filename: Name of the open boundaries conditions (T, S, $\bar u$, $\bar v$, $u$,
$v$). These files are covering only the open-boundaries slices, inducing files much
smaller than the ``climatology'' ones.    \\ 
&  \\ 

\large{\textbf{initial}} &     \\ 
&  NRREC: Record number of the restartfile to read as the initial conditions.    \\ 
& /filename: Name of the file containing initial state.\\ 
&  \\ 



\large{\textbf{restart}} &     \\ 
%\hline
&   NRST:      Frequency of writing\\ 
&   NRPFRST   0: writing several records every NRST time steps. -1: overwriting record
every NRST time steps\\ 
&  /filename  \\ 
&  \\ 

\large{\textbf{history}} &     \\ 
%\hline
&  LDEFHIS: flag (T/F) if writing history files  \\ 
&  NWRT: Frequency of writing \\ 
&  NRPFHIS: 0: writing several records every NWRT time steps. -1: overwriting record
every NRST time steps   \\ 
&  /filename: Name of the gistory file  \\ 
&  \\ 


\large{\textbf{averages}} &     \\ 
%\hline
&   NTSAVG: Starting timestep for the accumulation of output time-averaged data. For
instance, you might want to average over the last day of a thirty-day run.  \\ 
&   NAVG: frequency of writing \\ 
&   NRPFAVG: 0: writing several records every NWRT time steps. -1: overwriting record
every NAVG time steps  \\ 
&  /filename  \\ 
&  \\ 


\large{\textbf{primary\_history \_fields}} &     \\ 
&   Flags of written primary variables in history netCDF file \\ 
&  \\ 

\large{\textbf{auxiliary\_history \_fields}} &     \\ 
&  Flags of written auxiliary variables in history netCDF file \\ 
&  \\ 

\large{\textbf{primary\_averages}} &     \\ 
&  Flags of written primary variables in history netCDF file \\ 
&  \\ 

\large{\textbf{auxiliary\_averages}} &     \\ 
&  Flags of written variables in average netCDF file\\ 
&  \\ 

\large{\textbf{rho0}} & Mean density used in the Boussinesq equation. \\ 
&  \\  

\large{\textbf{lateral\_visc}} &     \\ 
&  VISC2: Laplaplacian background viscosity \\
&  VISC4: Bilaplacian  background viscosity \\
&  \\ 

\large{\textbf{tracer\_diff2}} &     \\ 
& TNU2(1:NT): Laplacian background diffusivity for  each tracer.  \\
&  \\ 

\large{\textbf{tracer\_diff4}} &     \\ 
& TNU4(1:NT): Laplacian background diffusivity for  each tracer.  \\
&  \\ 

\large{\textbf{vertical\_mixing}} &     \\ 
& Coefficient in case of use of analytical vertical mixing scheme.  \\ 
&  \\ 

\large{\textbf{bottom\_drag}} &     \\ 
& RDRG [m/s]: Drag coefficient in case of linear bottom stress formulation. \\
& \\
& RDRG2: Drag coefficient in case of constant quadratic bottom stress formulation. \\
& \\
& Zob [m]: Rugosity length in case of Von-Karman quadratic bottom stress formulation \\
& Cdb\_min: Minimum value of the drag coefficient in case Von-Karman quadratic bottom stress formulation. \\
& Cdb\_max: Maximum value of the drag coefficient in case Von-Karman quadratic bottom stress formulation. \\
&  \\ 

\large{\textbf{gamma2}} &     \\ 
&  Free slip boundary condition. 1 mean free slip condition are ON. \\
&  \\ 
%\hline
\large{\textbf{sponge}} &     \\
&     X\_SPONGE [m]: widthness of the sponge layers.\\
&     V\_SPONGE [$m^2/sec$]: Value of the viscosity and diffusity enhanced value at
the boundary point in the nudging/sponge layer. These value are enhanced following a
linear profil in the sponge/nudging layer, from the interior value to the max value
V\_SPONGE at boundary.\\
&  \\ 

%\hline
\large{\textbf{nudg\_cof}} &     \\
& TauT\_in  [days]: Nudging time scale for tracer signal going inward the
domain. This coefficient is used at boundary point and impose a
strong nudging towards climatology external data.  \\
& \\
& TauT\_out [days]: Nudging time scale for tracer signal going outward the domain. This coefficient is used at boundary point and impose a
smooth nudging towards climatology external data. This coefficient is also used in
the nudging/sponge layer to add a smooth nudging towards data. This is on only if the
CLIMATOLOGY boundary stategy is used, not the BRY one.\\
& \\
& TauM\_in  [days]: Same as above, but concerning the momentum equations. \\
& \\
& TauM\_out [days]: Same as above, but concerning the momentum equations. \\
&  \\ 

%\hline
\large{\textbf{diagnostics}} &     \\
&ldefdia: Boolean flag to activate the tracer equation "snap-shot" diagnostic file writing \\
&nwrtdia: Frequency of writing     \\
&nrpfdia: nrpfdia: 0: writing several records every nwrtdia time steps. -1: overwriting record
every nwrtdia time steps   \\
&/filename: Name of the file tracer equation diagnostic file.\\

\large{\textbf{diagnostics\_history \_fields}} &     \\
& Flags of written tracer equation term in history netCDF file \\
& \\


\large{\textbf{diag\_avg}} &     \\
&ldefdia\_avg: Boolean flag to activate the tracer equation average diagnostic file writing  \\
&ntsdia\_avg: Starting timestep for the accumulation of output time-averaged data. For
instance, you might want to average over the last day of a thirty-day run.  \\
&nwrtdia\_avg: Frequency of writing and average time.   \\
&nprfdia\_avg: nrpfdia\_avg: 0: writing several records every nwrtdia\_avg time steps. -1: overwriting record
every nwrtdia\_avg time steps \\
&/filename: Name of the file tracer equation average diagnostic file. \\
&  \\ 

\large{\textbf{diagnostics\_average \_fields}} &     \\
& Flags of written tracer equation term in average netCDF file\\
& \\

\large{\textbf{diagnosticsM}} &   Same format as \large{\textbf{diagnostics}} but for the momentum equation. \\
&ldefdiaM:     \\
&nwrtdiaM:     \\
&nrpfdiaM:     \\
&/filename:    \\
&  \\ 

\large{\textbf{diagnosticsM \_history\_fields}} &     \\
& Flags of written momentum equations terms in history netCDF file\\
& \\

\large{\textbf{diagM\_avg}} &  Same format as \large{\textbf{diag\_avg}} but for the momentum equation. \\
&ldefdiaM\_avg:     \\
&ntsdiaM\_avg:      \\
&nwrtdiaM\_avg:     \\
&nprfdiaM\_avg:     \\
&/filename          \\
&  \\ 

\large{\textbf{diagnosticsM \_average\_fields}} &     \\
& Flags of written momentum equations terms in average netCDF file\\
& \\


\large{\textbf{diagnosticsM\_bio}} & Same format as \large{\textbf{diagnostics}}     \\
&ldefdiabio   \\
&nwrtdiabio   \\
&nrpfdiabio   \\
&/ filename    \\
&  \\ 

\large{\textbf{diagbio\_avg}} & Same format as \large{\textbf{diag\_avg}}  \\
&ldefdiabio\_avg  \\
&ntsdiabio\_avg   \\
&nwrtdiabio\_avg  \\
&nprfdiabio\_avg  \\
&/ filename \\
&  \\ 

\large{\textbf{biology}} &     \\
& Name of the file containing the Iron dust forcing if using PISCES biogeochemical
mode. \\
&  \\ 

\large{\textbf{sediments}} &     \\
&  \\ 

\large{\textbf{sediment\_history \_fields}} &     \\
& bed\_thick  \\
& bed\_poros \\ 
& bed\_fra(sand,silt) \\
&  \\ 

\large{\textbf{bbl\_history\_fields}} &     \\
& Abed  \\
& Hripple  \\
& Lripple  \\
& Zbnot  \\
& Zbapp  \\
& Bostrw \\
&  \\ 

\large{\textbf{floats}} & Lagrangian floats application \\
& Same format as \large{\textbf{diagnostics}}  \\
& LDEFFLT \\
& NFLT \\
& NRPFFLT  \\
& / inpname, hisname \\
&  \\ 

\large{\textbf{floats\_fields}} &     \\
& Type of fields computed for each lagrangian floats. \\
&  \\ 

\large{\textbf{station\_fields}} &  Fixed station application.   \\
& Same format as \large{\textbf{diagnostics}}  \\
& LDEFSTA \\
& NSTA    \\
& NRPFSTA \\
&/ inpname, hisname \\
&    \\

\large{\textbf{psource}} &     \\
& Nsrc   \\
& Isrc   \\
& Jsrc   \\
& Dsrc   \\
& Qbar [m3/s]   \\  
& Lsrc   \\      
& Tsrc   \\
&  \\ 
\hline
\caption{Description of the roms.in file}
\label{ta:desc_roms.in}
\end{longtable}

%%%%%%%%%%%%%%%%%%%%%%%%%%%%%%%%%%%%%%%%%%%%%%%%%%%%%%%%%%%%%%%%%%%%%%%%%%%%%%%
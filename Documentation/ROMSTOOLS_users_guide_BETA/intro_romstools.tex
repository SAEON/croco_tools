\section{Installation}
\subsection{System requirement}
This toolbox has been designed for Matlab. It needs at least 
2 Gbites of disk space. It has been tested on several 
Matlab versions ranging from Matlab6 to Matlab2006a. It has been 
mostly tested on Linux workstations, but it could be used 
on any platform if a NetCDF and a LoadDAP  Matlab Mex files  
are provided. The NetCDF Matlab Mex file is needed to read 
and write into NetCDF files and it can be found at the web 
location: http://mexcdf.sourceforge.net/. The LoadDAP Matlab Mex file
is used to download data from OpenDAP servers for inter-annual and forecast 
simulations. It can be found at the web location: 
http://www.opendap.org/download/ml-structs.html. The Matlab 
LoadDAP Mex file provides a way to read any OpenDAP-accessible 
data into Matlab. Note that the LibDAP library must be installed
on your system before installing LoadDAP. Details can be found 
at the web location: http://www.opendap.org. MexCDF and LoadDAP mex 
files are provided for Linux (system FEDORA 32bits: mexcdf and 
Opendap\_tools/FEDORA ; system CENTOS or FEDORA 64bits: 
mexnc and Opendap\_tools/FEDORA\_X64), but they are not working 
on all the plateforms.

All the other necessary Matlab toolboxes (i.e. air-sea, mask, 
netcdf or m\_map...) are included in the ROMSTOOLS package. 
Global datasets, such as topography \citep{Smi97}, 
hydrography \citep{Con02} or surface fluxes \citep{Das94}, are 
also included.

\subsection{Getting the files}

All the necessary compressed tar files (XXX.tar.gz) containing 
the Matlab programs, several datasets  and other toolboxes and 
softwares needed by ROMSTOOLS are located at:
\begin{center}
http://roms.mpl.ird.fr/Roms\_tools/index.html 
\end{center}
For the ROMS source code you should download ROMS\_AGRIF version
V1.1.\\\\\\\\\\\\


\subsection{Extracting the files}

Download all the compressed tar files. Uncompress and untar all 
the files (gunzip and tar -xvf). You should obtain the following 
directory tree : \\
\\
Roms\_tools

$|$-- Aforc\_NCEP

$|$-- Aforc\_QuikSCAT

$|$-- air\_sea 

$|$-- COADS05  

$|$-- Compile

$|$-- Diagnostic\_tools

$|$-- Documentation

$|$\hspace{0.5cm}$|$-- User\_guide

$|$-- Forecast\_tools

$|$-- mask

$|$-- mex60

$|$-- mexnc

$|$\hspace{0.5cm}$|$-- tests

$|$-- m\_map

$|$\hspace{0.5cm}$|$-- private

$|$-- Nesting\_tools

$|$-- netcdf\_g77

$|$-- netcdf\_ifc

$|$-- netcdf\_matlab

$|$\hspace{0.5cm}$|$-- listpick

$|$\hspace{0.5cm}$|$-- ncatt

$|$\hspace{0.5cm}$|$-- ncbrowser

$|$\hspace{0.5cm}$|$-- ncdim

$|$\hspace{0.5cm}$|$-- ncfiles

$|$\hspace{0.5cm}$|$-- ncitem

$|$\hspace{0.5cm}$|$-- ncrec

$|$\hspace{0.5cm}$|$-- nctype

$|$\hspace{0.5cm}$|$-- ncutility

$|$\hspace{0.5cm}$|$-- ncvar

$|$\hspace{0.5cm}$|$-- netcdf

$|$-- netcdf\_x86\_64

$|$-- Oforc\_OGCM

$|$-- Opendap\_tools

$|$\hspace{0.5cm}$|$-- FEDORA

$|$\hspace{0.5cm}$|$-- FEDORA\_X64

$|$-- Preprocessing\_tools

$|$-- Roms\_Agrif

$|$\hspace{0.5cm}$|$-- AGRIFZOOM

$|$\hspace{0.5cm}$|$\hspace{0.5cm}$|$-- AGRIF\_FILES

$|$\hspace{0.5cm}$|$\hspace{0.5cm}$|$-- AGRIF\_INC

$|$\hspace{0.5cm}$|$\hspace{0.5cm}$|$-- AGRIF\_OBJS

$|$\hspace{0.5cm}$|$\hspace{0.5cm}$|$-- AGRIF\_YOURFILES

$|$\hspace{0.5cm}$|$\hspace{0.5cm}$|$-- LIB.clean

$|$-- Run

$|$\hspace{0.5cm}$|$-- DATA


$|$\hspace{0.5cm}$|$-- FORECAST

$|$\hspace{0.5cm}$|$-- ROMS\_FILES

$|$\hspace{0.5cm}$|$-- SCRATCH

$|$\hspace{0.5cm}$|$-- TEST\_CASES

$|$-- SeaWifs

$|$-- SST\_pathfinder

$|$-- Tides

$|$-- Topo

$|$\hspace{0.5cm}$|$-- Matlab

$|$-- TPX06

$|$-- TPX07

$|$-- Visualization\_tools

$|$-- WOA2001

$|$-- WOA2005
\\
\\
Definition of the different directories :
\begin{itemize}
\item Aforc\_NCEP : Scripts for the recovery of surface forcing data 
      (based on NCEP reanalysis) for inter-annual simulations.
\item Aforc\_QuikSCAT : Scripts for the recovery of wind stress from 
      satellite scatterometer data (QuickSCAT).
\item COADS05 : Directory of the surface fluxes global monthly 
      climatology at $0.5^\circ$ resolution \citep{Das94}.
\item Compile : Empty scratch directory for ROMS compilation.
\item Diagnostic\_tools : A few Matlab scripts for animations and
      basic statistical analysis.
\item Documentation : Location of the ROMSTOOLS user guide.
\item Forecast\_tools : Scripts for the generation of an operational
      modeling system 
\item mask : Land mask edition toolbox developed by A.Y. Shcherbina.
\item mex60 : Matlab NetCDF interface for 32 bits Linux architectures.
\item mexnc : Matlab NetCDF interface for 64 bits Linux architectures.
%\item \textbf{mexnc32} : 
%\item \textbf{mexnc64} : 
\item m\_map : The Matlab mapping toolbox 
      (http://www2.ocgy.ubc.ca/$\sim$rich/map.html).
\item Nesting\_tools : Preprocessing tools used to prepare nested
      models.
\item netcdf\_g77 : The NetCDF Fortran library for Linux, compiled using g77\\
      (http://www.unidata.ucar.edu/packages/netcdf/index.html).
\item netcdf\_ifc : The NetCDF Fortran library for Linux, compiled with ifort. 
      The Intel Fortran Compiler (ifort) is available at \\
      http://www.intel.com/software/products/compilers/flin/noncom.htm.
\item netcdf\_matlab : The Matlab NetCDF toolbox  \\
      ({\small
      http://woodshole.er.usgs.gov/staffpages/cdenham/public\_html/MexCDF/nc4ml5.html}).
\item netcdf\_x86\_64 : The NetCDF Fortran library for Linux, compiled with ifort
      on a 64 bits architecture.
\item Oforc\_OGCM : Scripts for the recovery of initial and lateral boundary 
      conditions from global OGCMs (SODA \citep{Car05} or ECCO \citep{Sta99}) for 
      inter-annual simulations.
\item Opendap\_tools : LoadDAP mexcdf and several scripts to automatically
      download data over the Internet.
\item Preprocessing\_tools : Preprocessing Matlab scripts (make\_grid.m, 
      make\_forcing, etc...).
\item Roms\_Agrif : ROMS Fortran sources.
\item Run : Working directory. This is where the ROMS input files
      are generated and where the model is running.
\item SeaWifs : surface chlorophyll-a climatology based on SeaWifs observations.
\item SST\_pathfinder : Directory of a higher resolution SST climatology 
      \citep{Rey94} for the thermal correction term.
\item Tides : Matlab routines to prepare ROMS tidal simulations. Tidal data
      are derived from the Oregon State University global models of ocean tides 
      TPXO6 and TPXO7 \citep{Egb02}: 
      http://www.oce.orst.edu/research/po/research/tide/global.html.
\item Topo : Location of the global topography dataset at $2'$ resolution
      \citep{Smi97}. Original data can be found at:
      http://topex.ucsd.edu/cgi-bin/get\_data.cgi
\item TPX06 : Directory of the global model of ocean tides TPXO6 \citep{Egb02}.
\item TPX07 : Directory of the global model of ocean tides TPXO7 \citep{Egb02}.
\item Visualization\_tools : Matlab scripts for the ROMS visualization
      graphic user interface.
\item WOA2001 : World Ocean Atlas 2001 global dataset 
      (monthly climatology at $1^\circ$ resolution) \citep{Con02}.
\item WOA2005 : World Ocean Atlas 2005 global dataset 
\item WOAPISCES : World Ocean Atlas Global dataset for biogeochemical PISCES data

\end{itemize}



\section{LibDAP and  LoadDAP}
It is sometimes difficult to compile LoadDAP.  LibDAP must be installed before
installing LoadDAP. You have to declare the LibDAP binary and library in tour
~/.bashrc with th command PATH and LD\_LIBRARY\_PATH. Once, compile and install
LoadDAP.

Here are a few instructions for the installation of these libraries:\\
\begin{itemize}
\item Download libDAP and loadDAP tar.gz version at the web location
http://www.opendap.org
\item To build the libDAP library, follow these steps:\\
\item log you as a root
\item Uncompress and untar the file libdap.tar.gz (gunzip and tar -xvf)
\item $>$: cd libdap\_directory
\item Type './configure' at the prompt. Some libraries must be installed 
on your system to successfully run configure and build libDAP library : libcurl 
(http://curl.haxx.se/) and libxml2 (http://xmlsoft.org/).\\
Example:\\
------------------------------------------------------------------------------------------\\
checking for a BSD-compatible install... /usr/bin/install -c\\
checking whether build environment is sane... yes\\
checking whether make sets (MAKE)... yes\\
checking build system type... i686-pc-linux-gnu\\
checking host system type... i686-pc-linux-gnu\\
checking for gawk... (cached) mawk\\
checking for g++... g++\\
checking for C++ compiler default output file name... a.out\\
...\\
config.status: dods-datatypes.h is unchanged\\
config.status: executing depfiles commands\\
------------------------------------------------------------------------------------------\\
\item Type 'make' to build the library.\\
Example :\\
------------------------------------------------------------------------------------------\\
make[1]: Entering directory '/home/tropic/tan/soft/libdap-3.6.2'\\
Making all in gl\\
make[2]: Entering directory '/home/tropic/tan/soft/libdap-3.6.2/gl'\\
make  all-am\\
make[3]: Entering directory '/home/tropic/tan/soft/libdap-3.6.2/gl'\\
...\\
------------------------------------------------------------------------------------------\\
\item Type 'make check' to run the tests. To pass this step you 
must have DejaGNU framework (GNU FTP mirror list:
http://www.gnu.org/prep/ftp.html).\\
Example :\\
------------------------------------------------------------------------------------------\\
make[1]: Entering directory `/home/tropic/tan/soft/libdap-3.6.2/gl'\\
dejagnu\_driver.sh\\
...\\
Test Run By tan on Thu Jul 19 11:19:02 2007\\
Native configuration is i686-pc-linux-gnu\\
                ===== das-test tests =====\\
		Running ...\\
                ===== das-test Summary =====\\		
                ===== dds-test tests =====\\
		Running ...\\
		===== dds-test Summary =====\\
                ===== expr-test tests =====\\
		Running ...\\
                ===== expr-test Summary =====\\
PASS: dejagnu\_driver.sh\\
==================\\
All 1 tests passed\\
==================\\
make[2]: Leaving directory `/home/tropic/tan/soft/libdap-3.6.2/tests'\\
make[1]: Leaving directory `/home/tropic/tan/soft/libdap-3.6.2/tests'\\
------------------------------------------------------------------------------------------\\
\item Type 'make install' to install the library. By default the files are installed under
/usr/local/lib/. You can specify a different root directory using the following control :
'make install root\_directory'.\\
\item Go to the .bashrc and add \\EXPORT LD\_LIBRARY\_PATH=\$LD\_LIBRARY\_PATH
: root\_directory.'\\

\item The installation of the loadDAP library is done as for libDAP. 
By default the files are installed under
/usr/local/share/.

\item Go in the .dodsrc file, add the PROXY\_SERVER configuration, if needed,
    for example PROXY\_SERVER http,proxy.legos.obs-mip.fr:$3128$
\end{itemize}

\section{Future plans}
\begin{itemize}
\item A graphic user interface could be useful for the preprocessing tools.
\item There is need for an improvement of the extrapolation and interpolation
methods.
\end{itemize}

\section{Warnings}
\begin{itemize}
\item Since Geostrophy is used to obtain the horizontal currents for
the lateral boundary conditions, this method should be applied with 
care close to the Equator. An extrapolation of the currents outside 
an equatorial band (2$^\circ$S-2$^\circ$N) is performed to get an 
approximation of the equatorial currents.
\item On extended grids, the objective analysis used for data 
extrapolation can be relatively costly in memory and CPU time. 
The "nearest" Matlab function that is less costly can be used instead.
If the computer starts to swap, you should think of reducing the 
dimension of your model's domain.
\end{itemize}
\newpage
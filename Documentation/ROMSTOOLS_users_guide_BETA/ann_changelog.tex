\section{Changelog since ROMS\_AGRIF $1.0$} \label{changelog}
\begin{itemize}
\item New diffusive-advection schemes : RSUP3 \citep{Marches09} \\

  To avoid unacceptable spurious diapycnal mixing, a new advection scheme has been
  proposed and validate : the RSUP3 scheme. The diffusion is split from advection and
  is represented by a rotated biharmonic diffusion scheme with flow-dependent
  hyperdiffusivity satisfying the Peclet constraint. The rotated diffusion operator
  is designed for numerical stability, which includes improvements of linear
  stability limits and a clipping method adapted to the sigma-coordinate.

  This scheme induce a time step smaller than the third-order upstream biaised
  diffusive advective scheme used in the version $1.0$. It is activated by the use of
  the cppkeys \textit{TS\_SPLIT\_UP3} for tracers and \textit{UV\_SPLIT\_UP3} for
  momentum in the cppdefs.h file.

  To avoid numerical instabilities in the sponge where there is enhanced
  diffusion/siffsuivity, a classical laplacian diffusion can be applied by the use of
  the cppkey \textit{SPONGE\_DIF2} and \textit{SPONGE\_VIS2} in the cppdefs.h file.





\item Two-ways AGRIF nesting : \\

As presented before, it is the capability of the fine grid to update data in the
coarse grid. With this procedure, we are now able to get the impact of high
resolution on the more coarser reolution, in a context of upscaling.

\item Online diagnostics and I/O : \\




\item PISCES biogeochemical model : \\





\end{itemize}



\paragraph { New advection-diffsion schemes \\}
\subsection{Numerical issues}
\subsubsection{Advection-Diffusion schemes}\label{sec:split3d}
...r�f�rence a votre papier Patrick et Laurent

\subsubsection{2D-3D coupling}\label{sec:2d-3d-coupling}
...reference au nouveau stepping ....??

\subsubsection{MPI paralleization}\label{sec:2d-3d-coupling}
... effort de parrallelisation MPI



\subsection{Nesting}\label{sec:2d-3d-coupling}
... r�f�rence au 2-ways nesting et au futur papier :-) uswc 5-15 



\subsection{Online diagnostics and I/O} \label{sec:online-diagn-io}
On parle ici de 
\begin{itemize}
\item Nouvelles variables sorties
\item Diagnostics TS MLD/ADV
\item Nouvelles cles pour les bulk
\end{itemize}




\subsection{PISCES biogeochemical model}\label{sec:pisces}
... ref�rence au possibilit�es de simulation avec le modele PISCES





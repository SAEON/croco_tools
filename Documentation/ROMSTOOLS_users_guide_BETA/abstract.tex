% This guide presents a series of Matlab routines which could be useful for the pre-
% and post-processing of oceanic regional ROMS simulations.

% This report is a basic users guide presenting the details of the methods and datasets
% are described in \citet{Pen07} and some

The Regional Ocean Modeling System (ROMS) is a new generation ocean circulation model
\citep{Shc03b} that has been specially designed for accurate simulations of regional
oceanic systems.  The reader is referred to \citet{Shc03a} and to \citet{Shc03b} for
a complete description of the model.  ROMS has been applied for the regional
simulation of a variety of different regions of the world oceans
\citep[e.g.][]{Bla02,Dil03,Hai00,Mac02,Mar03,Pen01}. \\

To perform a regional simulation using ROMS, the modeler needs to provide several
data files in a specific format: horizontal grid, bottom topography, surface forcing,
lateral boundary conditions... He also needs to analyze the model outputs. The tools
which are described here have been designed to perform these tasks.  The goal is to
be able to build a standard regional model configuration in a minimum time. \\

In the first chapter, the system requirements and the installation process are
exposed. A short note on ROMS model is presented in chapter 2. A tutorial on the use
of ROMSTOOLS is shown in the third section. Tidal simulations, inter-annual
simulations, nesting tools, biology and operational regional modeling are presented
in section 4,
5, 6, 7 and 8. \\

In the second chapter, some details of the IRD version of ROMS new release, named
roms\_agrif $2.0$, using the AGRIF nesting procedure are presented. \\
First, the new AGRIF 2-ways nesting procedure implemented in the code is described,
then new numerical and physical schemes and parametrization are exposed. Then a
changelog section since the last roms\_agrif $1.0$ offical version is presented.
Finally, the cpp-keys, parameters and input files are described in details to
correctly configure the model options.


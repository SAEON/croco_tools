ROMSTOOLS can help to realize inter-annual simulations. In this context, 
we rely on Ocean Global Circulations Models (OGCM) for the lateral 
boundary conditions and a global atmospheric reanalysis for the surface 
forcing (NCEP). To limit the volume of data which needs to be transfered 
over the Internet, we use Opendap to extract only the necessary subgrids. 

\subsection{Getting the surface forcing data from NCEP}

The Matlab script make\_NCEP.m is used to obtain the surface forcing data.
It downloads the necessary NCEP surface forcing data (Sea Surface 
Temperature, Wind stress ...) over the Internet, and interpolates them on 
the model grid. Since make\_NCEP.m works with the bulk parameterization 
(i.e. the 
BULK\_FLUX and BULK\_EP cpp keys should be defined in cppdefs.h),
a surface forcing NetCDF file and a bulk NetCDF file are generated for 
each month of your simulation in the directory 
$\sim$/Roms\_tools/Run/ROMSFILES/ .
The part of the file romstools\_param.m that you should change is:\\\\
\%\\
\%\%\%\%\%\%\%\%\%\%\%\%\%\%\%\%\%\%\%\%\%\%\%\%\%\%\%\%\%\%\%\%\%\%\%\%\%\%\%\%\%\%\%\%\%\%\%\%\%\%\%\%\%\%\%\%\%\%\%\%\%\%\%\%\%\%\%\%\%\%\\
\%\\
\% 7 Parameters for Interannual forcing (SODA, ECCO, NCEP, ...)\\
\%\\
\%\%\%\%\%\%\%\%\%\%\%\%\%\%\%\%\%\%\%\%\%\%\%\%\%\%\%\%\%\%\%\%\%\%\%\%\%\%\%\%\%\%\%\%\%\%\%\%\%\%\%\%\%\%\%\%\%\%\%\%\%\%\%\%\%\%\%\%\%\%\\
\%\\
\% Path to Forcing data\\
\%\\
FORC\_DATA\_DIR = [RUN\_dir,'DATA/'];\\
\%\\
Download\_data = 1;                            \% Get data from the OPENDAP sites \\
level         = 0;                            \% AGRIF level; 0=parent grid \\
\%\\
\%  Options for make\_NCEP\\
\%\\
NCEP\_dir= [FORC\_DATA\_DIR,'NCEP\_',ROMS\_config,'/']; \% NCEP data directory\\
makefrc      = 1;                            \% 1: Create forcing files\\
makeblk      = 1;                            \% 1: Create bulk files\\
add\_tides     = 0;                     \% 1: Add the tides (To be done...)\\
\%\\
NCEP\_version  = 1;                            \% NCEP version:\\
\%                                 (1: NCEP/NCAR Reanalysis, 1/1/1948 - present\\
\%                                  2: NCEP-DOE Reanalysis, 1/1/1979 - 12/31/2001)\\
\%\\
\\
Variables description :
\begin{itemize}
\item FORC\_DATA\_DIR : Directory where the different files downloaded over 
the Internet are stored.
\item Download\_data : Get data from the OPENDAP sites. Should be 1.
\item level : AGRIF level. The parent grid = 0 and the child grid = 1.
\item NCEP\_dir= [FORC\_DATA\_DIR,'NCEP\_',ROMS\_config,'/'] : NCEP data directory. 
This is where NCEP data downloaded over the Internet are stored.
\item makefrc : Switch to define if the forcing file is generated. Should be 1.
\item makeblk : Switch to define if the bulk file is generated. Should be 1.
\item add\_tides : Switch to define if the tidal forcing is added. 
\item NCEP\_version : version of the NCEP reanalysis. 1: NCEP/NCAR Reanalysis, 1/1/1948 - present.
2: NCEP-DOE Reanalysis, 1/1/1979 - 12/31/2001.
\end{itemize}

Save romstools\_param.m and run make\_NCEP in the Matlab session.
You should obtain:
\\\\
$>>$ make\_NCEP\\
Add the paths of the different toolboxes\\
Arch : x86\_64 - Matlab version : 2006a\\
Use of mexnc and loaddap in 64 bits.\\
Download NCEP data with OPENDAP\\
\\
Get NCEP data from 2000 to 2000\\
Minimum Longitude: 12.3\\
Maximum Longitude: 20.3\\
Minimum Latitude: -35.5\\
Maximum Latitude: -26.3815\\
\\
Making output data directory ../Run/DATA/NCEP\_Benguela/\\
Process the first dataset: http://www.cdc.noaa.gov/cgi-bin/nph-nc/Datasets/ncep.reanalysis/surface\_gauss/\\
    Create ../Run/DATA/NCEP\_Benguela/land.sfc.gauss.nc\\
Processing year: 2000\\
  Processing month: 1\\
    Get air for year 2000 - month 1\\
...

\subsection{Getting the lateral boundary conditions}

Initial conditions and lateral boundary conditions and  can be 
obtained from several ocean global circulation models (OGCM) 
such as SODA \citep{Car05} or ECCO \citep{Sta99}. The SODA 
reanalysis is available from 1958 to 2001 and ECCO is available 
from 1993 until now. The Matlab script make\_OGCM.m is used to 
download data over the Internet, and to perform the interpolations 
on the model grid. 
A lateral boundary conditions NetCDF file is generated for each month 
of your simulation in the directory $\sim$/Roms\_tools/Run/ROMSFILES/ . 
\\
\\
The part of the file romstools\_param.m that you should change is:
\\
\\
\%\%\%\%\%\%\%\%\%\%\%\%\%\%\%\%\%\%\%\\
\%\\
\% Options for make\_OGCM \\
\%\\
\%\%\%\%\%\%\%\%\%\%\%\%\%\%\%\%\%\%\%\\
OGCM        = 'SODA';                                \% Select the OGCM:
SODA(1958-2001), ECCO(1993-2005), ...\\
OGCM\_dir    = [FORC\_DATA\_DIR,OGCM,'\_',ROMS\_config,'/'];  \\
bry\_prefix  = [ROMS\_files\_dir,'roms\_bry\_',OGCM,'\_']; \\
clm\_prefix  = [ROMS\_files\_dir,'roms\_clm\_',OGCM,'\_']; \\
ini\_prefix  = [ROMS\_files\_dir,'roms\_ini\_',OGCM,'\_']; \\
OGCM\_prefix = [OGCM,'\_'];                            \\
rmdepth     = 2;                                    \\ 
\%                             \\
\%                        \\
\\\\\\
Variables description :
\begin{itemize}
\item OGCM = 'SODA' : Name of the OGCM employed (SODA or ECCO).
\item OGCM\_dir    = [FORC\_DATA\_DIR,OGCM,'\_',ROMS\_config,'/']  : 
OGCM data directory.
\item bry\_prefix  = [ROMS\_files\_dir,'roms\_bry\_',OGCM,'\_'] : 
Left part of the boundary file name.
\item clm\_prefix  = [ROMS\_files\_dir,'roms\_clm\_',OGCM,'\_'] : 
Left part of the climatology file name.
\item ini\_prefix  = [ROMS\_files\_dir,'roms\_ini\_',OGCM,'\_'] : 
Left part of the initial file name.
\item OGCM\_prefix = [OGCM,'\_'] : 
Left part of the OGCM file name. This is where OGCM data are
stored. 
\item rmdepth = 2 : Number of bottom levels to remove.
This is useful when there is no valid data at this level.
For example, if the depth in the ROMS domain is shallower 
than the OGCM depth.
\end{itemize}
Save romstools\_param.m and run make\_OGCM in the Matlab session.
You should obtain:
\\\\
$>>$ make\_OGCM\\
Add the paths of the different toolboxes\\
Arch : x86\_64 - Matlab version : 2006a\\
Use of mexnc and loaddap in 64 bits.\\
Download data...\\
\\
Get data from Y2000M1 to Y2000M3\\
Minimum Longitude: 12.3\\
Maximum Longitude: 20.3\\
Minimum Latitude: -35.5\\
Maximum Latitude: -26.3815\\
\\
Making output data directory ../Run/DATA/SODA\_Benguela/\\
Process the dataset: http://iridl.ldeo.columbia.edu./SOURCES/.CARTON-GIESE/.SODA/.v1p4p3\\
Processing year: 2000\\
  Processing month: 1\\
    Download SODA for 2000 - 1\\
    ...SSH\\
    ...U\\
...

\subsection{Running the model}

Compile the model with jobcomp (and with the 
cpp keys BULK\_FLUX and BULK\_EP defined) and edit 
the input parameter file 
$\sim$/Roms\_tools/Run/roms\_inter.in as for the
climatology experiments. As for the long simulations, a csh script
(run\_roms\_inter.csh) manages the handling of input and output files.
It also changes the number of time steps so each month has the correct
length. This script takes care of leap years. For example Y1996M2 
(February 1996) is 29 days long.

Part to edit in run\_roms\_inter.csh:\\
\\
\#\\
set MODEL=roms\\
set SCRATCHDIR=`pwd`/SCRATCH\\
set INPUTDIR=`pwd`\\
set MSSDIR=`pwd`/ROMS\_FILES\\
set MSSOUT=`pwd`/ROMS\_FILES\\
set CODFILE=roms\\
set AGRIF\_FILE=AGRIF\_FixedGrids.in\\
\#\\
set BULK\_FILES=1\\
set FORCING\_FILES=1\\
set CLIMATOLOGY\_FILES=0\\
set BOUNDARY\_FILES=1\\
\#\\
\# Atmospheric surface forcing dataset (NCEP, GFS,...)\\
\#\\
set ATMOS=NCEP\\
\#\\
\# Oceanic boundary and initial dataset (SODA, ECCO,...)\\
\#\\
set OGCM=SODA\\
\#\\
\# Model time step [seconds]\\
\#\\
set DT=5400\\
\#\\
\# number total of grid levels (1: No child grid)\\
\#\\
set NLEVEL=1\\
\#\\
set NY\_START=2000\\
set NY\_END=2000\\
set NM\_START=1\\
set NM\_END=3\\
\#\\
\#  Restart file - RSTFLAG=0 --$>$ No Restart\\
\#		  RSTFLAG=1 --$>$ Restart\\
\#\\
set RSTFLAG=0\\
\#\\
\#  Time Schedule  -  TIME\_SCHED=0 --$>$ yearly files\\
\#                    TIME\_SCHED=1 --$>$ monthly files\\
\#\\
set TIME\_SCHED=1\\
\#\\
\#\#\#\#\#\#\#\#\#\#\#\#\#\#\#\#\#\#\#\#\#\#\#\#\#\#\#\#\#\#\#\#\#\#\#\#\#\#\#\#\#\#\#\#\#\#\#\#\#\#\#\#\#\#\#\#
\\
\\
Variables definitions:
\begin{itemize}
\item MODEL=roms : Name used for the input files. For example roms\_grd.nc.
\item SCRATCHDIR=`pwd`/SCRATCH : Scratch directory where the model is run.
\item INPUTDIR=`pwd` : Input directory where the roms\_inter.in input file
is located.
\item MSSDIR=`pwd`/ROMS\_FILES : Directory where the roms input NetCDF files
(roms\_grd.nc, roms\_frc.nc, ...) are stored.
\item MSSOUT=`pwd`/ROMS\_FILES : Directory where the roms output NetCDF files
(roms\_his.nc, roms\_avg.nc, ...) are stored.
\item CODFILE=roms : ROMS executable.
\item AGRIF\_FILE=AGRIF\_FixedGrids.in : AGRIF input file which defines the 
position of child grids when using embedding.
\item BULK\_FILES=1 : 1 if using bulk NetCDF files (should be 1 for NCEP).
\item FORCING\_FILES=1 : 1 if using forcing NetCDF files (should be 1 for NCEP).
\item CLIMATOLOGY\_FILES=0 : 1 if using XXX\_clm.nc files. Using a climatology
file for each month can take a lot of disc space. It is less costly to use
 boundary files (XXX\_bry.nc). 
\item BOUNDARY\_FILES=1 : 1 if using XXX\_bry.nc files. 
\item ATMOS=NCEP : name of the atmospheric reanalysis. For the moment it is only
NCEP.
\item OGCM=SODA : name of the OGCM for the boundary conditions. SODA or ECCO.
\item DT=5400 : Model time step in seconds.
\item NDAYS = 30 : Number of days in 1 month.
\item NLEVEL=1 : Total number of model grids (no embedding: NLEVEL=1).
\item NY\_START=2000 : Starting year.
\item NY\_END=2000 : Ending Year.
\item NM\_START=1 : Starting month.
\item NM\_END=3 : Ending month.
\item RSTFLAG=0 : 1 if restarting a simulation
\item TIME\_SCHED=1 : (obsolete) 0 if using yearly files, 1 if using monthly 
files. Since make\_NCEP and make\_OGCM are creating only monthly
files, it should be always 1.
\end{itemize}


As for ROMS long climatology experiments, inter-annual experiments can be run
in batch mode:\\
$>$: nohup ./run\_roms\_inter.csh $>$ exp1.out \&\\\\

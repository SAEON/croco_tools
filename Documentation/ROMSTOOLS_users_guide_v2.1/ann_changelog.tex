\section{Changelog since ROMS\_AGRIF 1.1} \label{changelog}
\begin{itemize}
\item \underline{New diffusive-advection schemes} : \textbf{TS\_SPLIT\_UP3}, Ref
  \citep{Marches09}

  To avoid unacceptable spurious diapycnal mixing, a new advection scheme has been
  proposed and validate : the RSUP3 scheme. The diffusion is split from advection and
  is represented by a rotated biharmonic diffusion scheme with flow-dependent
  hyperdiffusivity satisfying the Peclet constraint. The rotated diffusion operator
  is designed for numerical stability, which includes improvements of linear
  stability limits and a clipping method adapted to the sigma-coordinate.

  This scheme induce a time step smaller than the third-order upstream biaised
  diffusive advective scheme used in the version $1.1$. It is activated by the use of
  the cppkeys TS\_SPLIT\_UP3 cppdefs.h file.

  To avoid numerical instabilities in the sponge where there is enhanced
  diffusion/diffsuivity, a classical laplacian diffusion can be applied by the use of
  the cppkey SPONGE\_DIF2 and SPONGE\_VIS2 in the cppdefs.h file.


\item \underline{Two-way AGRIF nesting} : \textbf{AGRIF\_2WAY}, Ref. \citep{DebreuMarchesiello2010}

  As presented before, it is the capability of the fine grid to update data in the
  coarse grid. With this procedure, we are now able to get the impact of high
  resolution on the more coarser reolution, in a context of upscaling.  To activate
  the two ways nesting, you need to define the AGRIF and
  AGRIF\_2WAY cpp-keys.


\item \underline{New bulk formulation} : \textbf{BULK\_FAIRALL}, Ref. \citep{fairall_l996}

  A new bulk formulae has been set-up in the code, to compute surface wind stress and
  surface net heat fluxes, as described in Fairall et al, 1996. \citep{Fairall96},
  the use of this bulk formulae is activated by the cpp keys BULK\_FAIRALL.

\item \underline{Online diagnostics and I/O}:

  New diagnostics and outputs are now available in the ROMS\_AGRIF $2.0$., as 
  wind stresses, windspeed, and heat fluxes (latent, sensible, long-wave and solar
  short wave). They can be written in the netCDF history and average files. Moreover,
  bottom boundary layer thickness and euphotic depth layer in case of biological
  experiments can be saved. For more information, you can refer at the section
  \ref{namelistdesc}.

  Improvmement have also been carried out on the tracer and momentum equation term
  diagnostics.  

  From now, the tracer equation terms are in \textit{dia.nc} and \textit{dia\_avg.nc}
  netcdf file with flags that permit to choose exactly the term
  you want to write in the NetCDF files.\\
  By default this diagnostics are written in a divergence flux form, ($\partial{u
    T}_{x}$, $\partial{v T}_{y}$, ...) but they can be written in an "advective form"
  : $u\partial{T}_{x}$, $v
  \partial{T}_{y}$, ... by using the \textbf{DIAGNOSTICS\_TS\_ADV} cpp key.
  Some term have been added : the term integrated over the mixed layer depth (cppkeys
  \textbf{DIAGNOSTICS\_TS\_MLD}. The mixed layer depth is computed online, from the
  closure module.
\\ \\
The differents term of the tracer equation, that can be diagnose,
for each tracer,  expressed in $C^{o}.s^{-1}$ ,  are:
  \begin{itemize}
   \item Time evolution term (called "rate" term)
   \item Zonal advective term
   \item Meridian advective term
  �\item Vertical advective term
   \item Horizontal mixing term
   \item Vertical mixing term
   \item Nudging $+$ Surface forcing term (called Tforc term) 
   \item In the case of use \textbf{DIAGNOSTICS\_TS\_MLD} some additional terms are
     diagnosted :
     \begin{itemize}
     \item Time evolution term integrated over the surface mixed layer depth (MLD hereafter)
     \item Zonal advective term integrated over the MLD
     \item Meridian advective term integrated over the MLD
    �\item Vertical advective term integrated over the MLD
     \item Horizontal mixing term integrated over the MLD
     \item Vertical mixing term integrated over the MLD
     \item Nudging $+$ Surface forcing term integrated over the MLD
     \item Entrainement term at the base of the mixed layer
     \end{itemize}
  \end{itemize}


  Concerning the momentum equation term, the differents term of the equation can be
  saved in history \textit{diaM.nc} and average \textit{diaM\_avg.nc} netCDF files.
  As the tracer, there are flags to choose exactly the term you want to write in the
  netCDF file. \\

  The differents terms of the momentum equation, expressed in $m.s^{-2}$, that can be diagnose (for u and v)
  are :
  \begin{itemize}
   \item Time evolution term (called "rate" term)
   \item Pressure gradient term
   \item Coriolis term
  �\item Zonal advective
   \item Meridian advective
   \item Vertical advective
   \item Horizontal mixing
   \item Vertical mixing
  \end{itemize}


\end{itemize}






\section{Parameters description : param.h}\label{paramdesc}

In this section, we present the more important parameters to configure your own
run. These parameters are : 
\begin{itemize}
\item \textbf{Test case or realistic run}: \\
...  \\
\#if defined BASIN $----->$ {\color{red}\textbf{\textit{Test case}}} \\
      parameter (LLm0$=60$,  MMm0$=50$,  N$=10$) \\
\#elif defined REGIONAL $----->$ {\color{red}\textbf{\textit{Realistic run}}} \\
\#$~~$  elif defined  BENGUELA\_LR \\
$~~~~~$      parameter (LLm0$=41$, MMm0$=42$,  N$=32$)  $!$ $<--$ BENGUELA\_LR \\
\#$~~$  else \\ 
$~~~~~$      parameter (LLm0=$39$,  MMm0$=32$,  N$=20$) \\
\#$~~$  endif \\
... \\
LLm0  
MMm0
N

\item \textbf{Grid size}:
  \begin{itemize}
\item LLm0: Dimension (ghost points included) in  the $\xi$ direction.
\item MMm0: Dimension (ghost points included) in  the $\eta$ direction.
\item N: Number of $\rho$-vertical points, in the vertical grid.
  \end{itemize}


\item \textbf{Parallelization}: \\
....\\
$!$ \\
$!$ Domain subdivision parameters: \\
$!$ ====== =========== =========== \\
$!$ NPP            Maximum allowed number of parallel threads; \\
$!$ NSUB\_X,NSUB\_E  Number of SHARED memory subdomains in XI- and \\
$!$                                                ETA-directions; \\
$!$ NNODES        Total number of MPI processes (nodes); \\
$!$ NP\_XI,NP\_ETA  Number of MPI subdomains in XI- and ETA-directions; \\
$!$

In the case of OpenMP parallelization, NPP is the number of cpu used in the
computation, in the case of MPI parallelization, it is equal to to NNODES.

AUTOTLING (implemented by Laurent Debreu): cppkeys that enable to compute the optimum
subdomains partition in terms of computation time.

\item \textbf{Tides}: \\
...�\\
\#if defined SSH\_TIDES || defined UV\_TIDES \\
$~~~~~~~~~$      integer Ntides$~~~~~$    \\
$~~~~~~~~~$      parameter (Ntides=8)$--->$ {\color{red}\textit{\textbf{Number of wave in
    the total tidal signal}}} \\
\#endif \\ 
... \\

\end{itemize}
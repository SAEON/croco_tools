An operating coastal modeling system can be designed following the 
assumption that large scale offshore dynamics are slow in comparison 
to the coastal system. The lateral boundary conditions are interpolated 
from the last available ECCO model outputs and are kept constant during
the ROMS simulation. ECCO model outputs are delayed by about two to four 
weeks, but we suppose that they are still relevant for the present large 
scale oceanic structure. The Global Forecast System (GFS) is used for the 
surface forcing. A first day of simulation is run in hindcast mode. This 
will provide the initial conditions for the next simulated day. 
Using GFS as surface forcing and ECCO for the lateral boundary conditions, 
a forecast of 7 days is conducted. A UNIX C-Shell script 
(~/Roms\_tools/Run/run\_roms\_forecast.csh) manages
data downloading, the hindcast and forecast simulations
and datas storage.
The script run\_roms\_forecast.csh starts Matlab in 
batch mode to download
with OPENDAP the lateral boundary conditions from ECCO and 
the surface forcing from GFS. It interpolates the data on ROMS 
grid and launches the hindcast and the forecast runs.

The script run\_roms\_forecast.csh should be edited to change the
directory pathways (HOME, RUNDIR, PATH, LD\_LIBRAIRY\_PATH, MATLAB,...).

The ROMS input files $\sim$/Roms\_tools/Run/roms\_hindcast.in and \\
$\sim$/Roms\_tools/Run/roms\_forecast.in should also be edited to change
the length of the time step and the number of time steps. 
The ROMS input file roms\_hindcast.in should be defined such as 
the hindcast run duration
is 1 day and a restart file is generated at the end of the hindcast run.

The script run\_roms\_forecast.csh can be relaunched everyday in batch mode 
using crontab.


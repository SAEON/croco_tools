ROMSTOOLS can help for the design of ROMS biogeochemical 
experiments. For the initial conditions and lateral boundary
conditions, WOA provides a seasonal climatology for nitrate
concentration and WOA or SeaWifs can be used to obtain a 
seasonal climatology of surface chlorophyll concentration.
Phytoplankton is estimated by a constant chlorophyll/phytoplankton 
ratio derived from previous simulations. Zooplankton is estimated
in a similar way. The part which should be edited by the user in 
romstools\_param.m is:\\
\\ 
\%\%\%\%\%\%\%\%\%\%\%\%\%\%\%\%\%\%\%\%\%\%\%\%\%\%\%\%\%\%\%\\
\%\\
\% Open boundaries and initial conditions parameters\\
\%   used by make\_clim.m, make\_biol.m, make\_bry.m\\
\%\%\%\%\%\%\%\%\%\%\%\%\%\%\%\%\%\%\%\%\%\%\%\%\%\%\%\%\%\%\%\\
\%\\
\% World Ocean Atlas directory (WOA2001 or WOA2005) \\
\%\\
woa\_dir=[ROMSTOOLS\_dir,'WOA2005/'];\\
\%\\
\% Surface chlorophyll seasonal climatology (WOA2001 or SeaWifs)\\
\%\\
chla\_dir=[ROMSTOOLS\_dir,'SeaWifs/'];\\
\%\\\\
Variables description :
\begin{itemize}
\item woa\_dir=[ROMSTOOLS\_dir,'WOA2005/'] : Directory where the World Ocean
Atlas 2005 climatology \citep{Con02} is located. The World Ocean
Atlas 2001 climatology can also be used.
\item chla\_dir=[ROMSTOOLS\_dir,'SeaWifs/'] : Directory of the surface 
chlorophyll seasonal climatology.
\end{itemize}
Run make\_biol in the Matlab session :\\
$>>$\\
$>>$ make\_biol\\\\
You should obtain :\\
-------------------------------------------------------------------------------\\
Add\_no3: creating variables and attributes for the OA file\\
Add\_no3: creating variables and attributes for the Climatology file\\
\\ 
 Ext tracers: Roa = 0 km - default value = NaN\\
 Ext tracers: horizontal interpolation of the annual data\\
 Ext tracers: horizontal interpolation of the seasonal data\\
time index: 1 of total: 4\\
time index: 2 of total: 4\\
time index: 3 of total: 4\\
time index: 4 of total: 4\\
 \\
 Vertical interpolations\\
 \\
 NO3...\\
 Time index: 1 of total: 4\\
 Time index: 2 of total: 4\\
 Time index: 3 of total: 4\\
 Time index: 4 of total: 4\\
 \\
 CHla...\\
Add\_chla: creating variable and attribute\\
...\\
Make a few plots...\\
-------------------------------------------------------------------------------\\\\
The cpp keys related to biology:
\begin{itemize}
\item BIO\_NChlPZD : Select a 5 components (Nitrate, Chlorophyll, Phytoplankton, 
Zooplankton, Detritus) biogeochemical model.
\item BIO\_N2ChlPZD2 : Select a 7 components (Nitrate, Ammonium, Chlorophyll, 
Phytoplankton, Zooplankton, Small Detritus, Large Detritus) biogeochemical model. 
\item BIO\_N2P2Z2D2 : Select a 8 components (Nitrate, Ammonium, Small  
Phytoplankton, Large Phytoplankton, Small Zooplankton, Large Zooplankton,
Small Detritus, Large Detritus) biogeochemical model. 
\item DIAGNOSTICS\_BIO : Define if writing out fluxes between the biological
components.\\
\end{itemize}

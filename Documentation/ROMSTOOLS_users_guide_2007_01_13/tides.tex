Using the method described by \citet{Fla76}, ROMS is able to propagate the 
different tidal constituents from its lateral boundaries. To do so, define  
the cpp keys TIDES,  SSH\_TIDES and UV\_TIDES and recompile the model 
using jobcomp. To work correctly, the model should use the \citet{Fla76} 
open boundary radiation scheme (cpp key OBC\_M2FLATHER defined).\\
The tidal components are added to the forcing file (XXX\_frc.nc)
by the Matlab program make\_tides.m.
Edit the file : $\sim$/Roms\_tools/Run/romstools\_param.m.
The part of the file that you should change is :\\
\\
\%\%\%\%\%\%\%\%\%\%\%\%\%\%\%\%\%\%\%\%\%\\
\%\\
\% 5-Parameters for tidal forcing\\
\%\\
\%\%\%\%\%\%\%\%\%\%\%\%\%\%\%\%\%\%\%\%\%\\
\%\\
\% TPXO file name (TPXO6 or TPXO7)\\
\%\\
tidename=[ROMSTOOLS\_dir,'TPXO6/TPXO6.nc'];\\
\%\\
\% Number of tides component to process\\
\%\\
Ntides=10;\\
\%\\
\% Chose order from the rank in the TPXO file :\\
\% "M2 S2 N2 K2 K1 O1 P1 Q1 Mf Mm"\\
\% " 1  2  3  4  5  6  7  8  9 10"\\
\%\\
tidalrank=[1 2 3 4 5 6 7 8 9 10];\\
\%\\
\% Compare with tidegauge observations\\
\%\\
lon0=18.37;\\
lat0=-33.91;   \% Cape Town location\\
Z0=1;          \% Mean depth of the tidegauge in Cape Town\\
\\

Variables definitions :
\begin{itemize}
\item tidename=[ROMSTOOLS\_dir,'TPXO6/TPXO6.nc'] : Location of the netcdf tidal dataset. 
This file is derived from the Oregon State University global model of ocean tides TPXO.6 
\citep{Egb02}.  Data sources can be found at \\
http://www.oce.orst.edu/po/research/tide/global.html.
It is also possible to use TPXO7.
\item Ntides=10 : Number of tidal components to process. Warning!
This value should be identical to the value of the parameter Ntides in param.h:
"parameter (Ntides=10)".
\item tidalrank=[1 2 3 4 5 6 7 8 9 10] : Order to select the different tidal components.
\item lon0=18.37;lat0=-33.91;Z0=1 : Location of a tidal gauge to compare the interpolated values
with observations. \\
\end{itemize}
An important aspect is the definition of time and especially the choice of a 
time origin. This is defined in $\sim$/Roms\_tools/Run/romstools\_param.m:
\\
\\
\%\%\%\%\%\%\%\%\%\%\%\%\%\%\%\%\%\%\%\%\%\%\%\%\%\%\%\%\%\%\%\%\%\%\%\%\%\%\%\%\\
\%\\
\% 6-Temporal parameters (used for make\_tides, make\_NCEP, make\_OGCM)\\
\%\\
\%\%\%\%\%\%\%\%\%\%\%\%\%\%\%\%\%\%\%\%\%\%\%\%\%\%\%\%\%\%\%\%\%\%\%\%\%\%\%\%\\
\%\\
Yorig         = 1900;               \% reference time for vector time\\
                                    \% in roms initial and forcing files\\
\%\\
Ymin          = 2000;               \% first forcing year\\
Ymax          = 2000;               \% last  forcing year\\
Mmin          = 1;                  \% first forcing month\\
Mmax          = 3;                  \% last  forcing month\\
\%\\
Dmin          = 1;                  \% Day of initialization\\
Hmin          = 0;                  \% Hour of initialization\\
Min\_min       = 0;                  \% Minute of initialization\\
Smin          = 0;                  \% Second of initialization\\
\%\\
SPIN\_Long     = 0;                  \% SPIN-UP duration in Years\\
\\\\
The origin of time (Yorig: 1 january of year Yorig) should be kept the same
for all the preprocessing and postprocessing steps.
Save romstools\_param.m and run make\_tides in the Matlab session:
\\
$>>$\\
$>>$ make\_tides\\\\
You should obtain :\\
------------------------------------------------------------------------------------------\\
Start date for nodal correction : 1-Jan-2000\\
Reading ROMS grid parameters ...\\
Tidal components : M2 S2 N2 K2 K1 O1 P1 Q1 Mf Mm \\
Processing tide : 1 of 10\\
...\\
------------------------------------------------------------------------------------------\\

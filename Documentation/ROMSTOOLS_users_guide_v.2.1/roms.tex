\section{The ROMS model}

ROMS solves the primitive equations in an Earth-centered rotating environment, based
on the Boussinesq approximation and hydrostatic vertical momentum balance. ROMS is
discretized in coastline- and terrain-following curvilinear coordinates.  ROMS is a
split-explicit, free-surface ocean model, where short time steps are used to advance
the surface elevation and barotropic momentum, with a much larger time step used for
temperature, salinity, and baroclinic momentum.  ROMS employs a special 2-way
time-averaging procedure for the barotropic mode, which satisfies the 3D continuity
equation \citep{Shc03b}.  The specially designed predictor-corrector time step
algorithm used in ROMS allows a substantial increase in the permissible time-step
size.

ROMS has been designed to be optimized on shared memory parallel computer
architectures such as the SGI/CRAY Origin 2000. Parallelization is done by two
dimensional sub-domains partitioning. Multiple sub-domains can be assigned to each
processor in order to optimize the use of processor cache memory. This allow
super-linear scaling when performance growth even faster than the number of CPUs.

The third-order, upstream-biased advection scheme implemented in ROMS allows the
generation of steep gradients, enhancing the effective resolution of the solution for
a given grid size \citep{Shc98}. Explicit lateral viscosity is null everywhere in the
model domain except in sponge layers near the open boundaries where it increases
smoothly close to the lateral open boundaries.

A non-local, K-profile planetary (KPP) boundary layer scheme \citep{Lar94}
parameterizes the unresolved physical vertical subgrid-scale processes.  If a lateral
boundary faces the open ocean, an active, implicit, upstream biased, radiation
condition connects the
model solution to the surroundings \citep{Mar01}. \\

%------------------------------------------------------------------------
More informations and model description can also be found in the SCRUM Manuel
\citep{hedstroem97} and a more recent User Manual on ROMS (Rutgers
version\footnote{http://myroms.org}), both written by Kate Hedstr�m\footnote{Kate
  Hedstr�m, University of Alaska Fairbanks, Center for Arctic Region Supercomputing
  Center, University of Alaska Fairbanks, } \citep{hedstroem2009} \footnote{Many
  thanks to Kate Hedstro�m for this work.}. These documents are available on the
ROMS\_AGRIF web site : http://roms.mpl.ird.fr in the documentation section.

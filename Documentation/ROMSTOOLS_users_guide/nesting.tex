%
% nesting
%

\subsection{Introduction}

To address the challenge of bridging the gap between near-shore and
offshore dynamics, a nesting capability has been added to ROMS
and tested for the California Upwelling System \citep{Pen04}.
The method chosen for embedded griding takes advantage of the AGRIF
(Adaptive Grid Refinement in Fortran) package \citep{Bla99,Deb00,
Deb03a,Deb03b}.
AGRIF is a Fortran 95 package for the inclusion of adaptive mesh refinement
features within a finite difference numerical model. One of
the major advantages of AGRIF in static-grid embedding is the ability to
manage an arbitrary number of fixed grids and an arbitrary number of
embedding levels.

\begin{figure}[htbp]
\centerline{\psfig{figure=nesting_fig1.eps,width=15cm}}
\caption{Temporal coupling between a parent and a child grid
for a refinement factor of 3.  The coupling is done at the baroclinic
time step.}
\label{fig:temp_coupling}
\end{figure}

A recursive integration procedure manages the time evolution for the
child grids during the time step of the parent grids
(Figure \ref{fig:temp_coupling}). In order to preserve the CFL 
criterion, for a typical
coefficient of refinement (say, a factor of 3 for a 5 km resolution
grid embedded in a 15 km grid), for each parent time step the child
must be advanced using a time step divided by the coefficient of
refinement as many time as necessary to reach the time of the parent
(Figure (\ref{fig:temp_coupling})).  For simple 2-level embedding, the
procedure is as follows:\\
\begin{enumerate}
\item Advance the parent grid by one parent time step.
\item Interpolate the relevant parent variables in space and time
to get the boundary conditions for the child grid.
\item Advance the child grid by as much child time steps as necessary
to reach the new parent model time.
\item Update point by point the parent model by averaging the more
accurate values of the child model (in the case of 2-way embedding).
\end{enumerate}
The recursive approach used in AGRIF
allows the specification of any number
of embedding level.

\subsection{Embedded (child) model preparation}

To run an embedded model, the user must provide the grid, the surface 
forcing and the initial conditions. To name the different files,
AGRIF employs a specific strategy: if the parent file names are of
the form: XXX.nc, the first child names will be of the form: 
XXX.nc.1, the second: XXX.nc.2, etc... 
This convention is also applied for the "roms.in" input files.

A graphic user interface (NestGUI) facilitates the generation of 
the different NetCDF files. Launch nestgui in the Matlab session 
(in the $\sim$/Roms\_tools/Run/ directory):
\\ \\
$>>$\\
$>>$ nestgui
\\ \\
A window pops up, asking for a "PARENT GRID" NetCDF file 
(Figure \ref{fig:nestgui1}). In our Benguela test case, you should select 
$\sim$/Roms\_tools/Run/ROMSFILES/roms\_grd.nc (grid file) and click "open".
The main window appears (Figure \ref{fig:nestgui2}).

\begin{figure}[!ht]
\centerline{\psfig{figure=nesting_select.eps,width=5cm}}
\caption{Entrance window of NestGUI}
\label{fig:nestgui1}
\end{figure}
 
\begin{figure}[!ht]
\centerline{\psfig{figure=nesting.eps,width=12cm}}
\caption{The NestGUI main window}
\label{fig:nestgui2}
\end{figure}

To generate the child model you should follow several steps:

\begin{enumerate}

\item To define the child domain, click "Define child" and create
the child domain on the main window. The size of the grid child
(Lchild and Mchild) is now visible. This operation can be redone 
until you are satisfied with the size and the position of the child 
domain. The child domain can be finely tuned using the imin, 
imax, jmin and jmax boxes. 
Be aware that the mask interpolation from the parent grid
to the child grid is not optimal close to corners. Parent/Child
boundaries should be placed where the mask is showing a straight
coastline. A warning will be given during the interpolation
procedure if this is not the case.

\item "Interp child" : It generates the child grid file. Before, 
you should select if you are using a new topography
("New child topo" button) for the child
grid or if you are just interpolating the parent topography
on the child grid. In the first case, you should defines
what topography file will be used (e.g. 
$\sim$/Roms\_tools/Topo/etopo2.nc or another dataset).
You should also define if you want the volume of the child grid 
to match the volume of the parent close to the parent/child 
boundaries ("Match volume" button, it should be "on" by default).
You should also define define the r factor \citep{Bec93} 
for topography smoothing ("r-factor", 0.25 is safe) and
the number of points to connect the child topography to the
parent topography ("n-band", it follows the relation 
$h_{new}=\alpha.h_{child} + (1-\alpha).h_{parent}$, 
where $\alpha$ is going from 0 to 1 in "n-band" points 
from the parent/child boundaries).
You should also select the child minimum depth ("Hmin",
it should be lower or equal to the parent minimum depth),
the maximum depth at the coast ("Hmax coast"), the 
number of selective hanning filter passes for the deep 
regions ("n filter deep") and the number of final 
hanning filter passes ("n filter final").

\item "Interp forcing": It interpolates the parent 
surface forcing  on the child grid. Select the parent forcing file
to be interpolated (e.g. $\sim$/Roms\_tools/Run/ROMSFILES/roms\_frc.nc). 
The child forcing file roms\_frc.nc.1 will be created. 
The parent surface fluxes are interpolated on the child grid. 
You can use "Interp bulk" if you are using a bulk formula.
In this case, the parent bulk file 
(e.g. $\sim$/Roms\_tools/Run/ROMSFILES/roms\_blk.nc) will be
interpolated on the child grid.

\item "Interp initial": It interpolates parent initial
conditions on the child grid. Select the parent initial file
(e.g. $\sim$/Roms\_tools/Run/ROMSFILES/roms\_ini.nc).
The child initial file 
(e.g. $\sim$/Roms\_tools/Run/ROMSFILES/roms\_ini.nc.1) 
will be created.
If the topographies are different between the parent and 
the child grids, the child initial conditions are 
vertically re-interpolated. In this case you should check 
if the options "vertical corrections" and "extrapolations"
are selected.
"Interp biology" can be used to interpolate
parent biological variables for biogeochemical experiments.
"Interp restart" generates a child restart file from 
a parent restart file 
(e.g. $\sim$/Roms\_tools/Run/ROMSFILES/roms\_rst.nc). 
This can be done to "hot start" a child model after the 
spin-up of the parent model.

\item You can click on "Create roms.in.*" to generate a
child input file (roms.in.1) from the parent input
file and click on "Create AGRIF\_FixedGrids.in" to 
generate a AGRIF\_FixedGrids.in file (the file which
defines the child grid position in the parent grid).

\end{enumerate}

"river" can be used to locate the river on the coast.
"Interp clim" can be useful to generate boundary conditions 
to test the child model alone. 

\subsection{Compiling and running the model}

The ROMS nesting procedure needs a Fortran 95 compiler. For Linux PCs,
the Intel Fortran Compiler (ifort) is available at \\
http://www.intel.com/software/products/compilers/flin/noncom.htm.
To be able to compile ROMS with ifort, you should change the corresponding 
comments in jobcomp. Define AGRIF in \\
$\sim$/Roms\_tools/Run/cppdefs.h.
Other cpp keys are related to AGRIF:
\begin{itemize}
\item AGRIF\_OBC\_EAST : Open eastern boundary for the child grids.
\item AGRIF\_OBC\_WEST : Open western boundary for the child grids.
\item AGRIF\_OBC\_SOUTH : Open southern boundary for the child grids.
\item AGRIF\_OBC\_NORTH : Open northern boundary for the child grids.
\item AGRIF\_STORE\_BAROT\_CHILD : Store ubar and vbar during the parent step for the
child boundary conditions.
\item AGRIF\_FLUX\_BC : Apply parent/child barotropic boundary conditions has 
fluxes.
\item AGRIF\_POLY\_DUAVG : Apply a third order polynomial temporal interpolation 
for DU\_avg1 and DU\_avg2.
\item AGRIF\_LOCAL\_VOLCONS : Enforce parent-child mass conservation.
\item AGRIF\_OBC\_M2FLATHER :  Activate Flather open boundary conditions for ubar and vbar
for the child model .
\item AGRIF\_OBC\_M2ORLANSKI : Activate 2D radiation open boundary conditions for ubar and vbar
for the child model.
\item AGRIF\_OBC\_M2SPECIFIED : Activate specified open boundary conditions for ubar and vbar
for the child model.
\item AGRIF\_OBC\_M2CHARACT :  Activate open boundary conditions based on characteristic methods 
for ubar and vbar
for the child model.
\item AGRIF\_OBC\_M3ORLANSKI : Activate 2D radiation open boundary conditions for u and v
for the child model.
\item AGRIF\_OBC\_M3SPECIFIED : Activate specified open boundary conditions for u and v
for the child model.
\item AGRIF\_OBC\_M3CHARACT : Activate open boundary conditions based on characteristic methods 
for u and v
for the child model.
\item AGRIF\_OBC\_TORLANSKI :  Activate 2D radiation open boundary conditions for tracers
for the child model .
\item AGRIF\_OBC\_TUPWIND : Activate upwind open boundary conditions for tracers
for the child model.
\item AGRIF\_OBC\_TSPECIFIED  : Activate specified open boundary conditions for tracers
for the child model.
\end{itemize}
The default definitions should be sufficient for most of the applications.
\\
\\
It is possible to edit the file AGRIF\_FixedGrids.in. 
This file contains the child grid positions
(i.e. imin,imax,jmin,jmax) and coefficients of refinement. A first line
gives the number of children grids per parent (if AGRIF\_STORE\_BAROT\_CHILD
is defined, only one child grid can be defined per parent grid). A second
line gives the relative position of each grid and the coefficient of refinement 
for each dimension. 
Edit the input files roms.in.1, roms.in.2 , etc... to define correctly the 
file names and the time steps. To run the model, simply type at the prompt:
roms roms.in.

To visualize the ROMS model outputs for different grid levels, 
change the value in the "child models" box
in roms\_gui.



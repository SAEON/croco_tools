This section presents the essential steps for preparing
and running a regional ROMS simulation. This is 
done following the example of a model of the
Southern Benguela at low resolution.
 
\subsection{Getting started}

Once the installation has been successful, launch a Matlab session
in the directory: $\sim$/Roms\_tools/Run. Run the start.m
script to set the Matlab paths for this session. The start.m script
also makes the difference between 32 bits and 64 bits Linux architectures
and adjusts the paths in consequence: \\
\\
$>$ : cd Roms\_tools/Run \\
$>$ : matlab \\
$<$ M A T L A B $>$...\\
$>>$ start \\
Add the paths of the different toolboxes... \\
Arch : x86\_64 - Matlab version : 12 \\
Use of mex60 and loaddap in 32 bits. \\
$>>$ \\
\\
You are now ready to create a new configuration.
It is important to respect the order of the following preprocessing 
steps: make\_grid, make\_forcing, make\_clim.
For all the preprocessing steps, there is only one file to edit : 
$\sim$/Roms\_tools/Run/romstools\_param.m .
This file contains the necessary parameters for the generation
of the ROMS input NetCDF files.
The first section in romstools\_param.m  defines the general parameters,
such as title, working directories or file names:
\\ \\
\%\%\%\%\%\%\%\%\%\%\%\%\%\%\%\%\%\%\%\%\%\%\\
\%\\
\% 1- General parameters\\
\%\\
\%\%\%\%\%\%\%\%\%\%\%\%\%\%\%\%\%\%\%\%\%\%\\
\%\\
\%  ROMS title names and directories\\
\%\\
ROMS\_title  = 'Benguela Test Model';\\
ROMS\_config = 'Benguela';\\
ROMSTOOLS\_dir = '../';\\
RUN\_dir=[ROMSTOOLS\_dir,'Run/'];\\
ROMS\_files\_dir=[RUN\_dir,'ROMS\_FILES/'];\\
\%\\
\% ROMS file names (grid, forcing, bulk, climatology, initial)\\
\%\\
grdname=[ROMS\_files\_dir,'roms\_grd.nc'];\\
frcname=[ROMS\_files\_dir,'roms\_frc.nc'];\\
blkname=[ROMS\_files\_dir,'roms\_blk.nc'];\\
clmname=[ROMS\_files\_dir,'roms\_clm.nc'];\\
ininame=[ROMS\_files\_dir,'roms\_ini.nc'];\\
oaname =[ROMS\_files\_dir,'roms\_oa.nc'];    \% oa file  : intermediate file not used\\
                                          \%            in roms simulations\\
bryname=[ROMS\_files\_dir,'roms\_bry.nc'];\\
Zbryname=[ROMS\_files\_dir,'roms\_bry\_Z.nc'];\% Zbry file: intermediate file not used\\
                                          \%            in roms simulations\\
\%\\
frc\_prefix=[ROMS\_files\_dir,'roms\_frc'];   \% generic bulk forcing file name \\
                                          \% for inter-annual roms simulations (NCEP or GFS)\\
blk\_prefix=[ROMS\_files\_dir,'roms\_blk'];   \% generic forcing file name\\
                                          \% for inter-annual roms simulations (NCEP or GFS)\\
\%\\
\% Objective analysis decorrelation scale [m]\\
\% (if Roa=0: simple extrapolation method; crude but much less costly)\\
\%\\
\%Roa=300e3;\\
Roa=0;\\
\%\\
interp\_method = 'cubic';           \% Interpolation method: 'linear' or 'cubic'\\
\%\\
makeplot     = 1;                 \% 1: create a few graphics after each preprocessing step\\
\%\\
\%\%\%\%\%\%\%\%\%\%\%\%\%\%\%\%\%\%\%\%\%\%\%\%\%\%\%\%\%\%\%\%\%\%\%\%\%\%\\
\\ 
Variables description:
\begin{itemize}
\item title='Benguela Test Model' : General title. You can give any name 
you want for your configuration.
\item ROMS\_config = 'Benguela' : Name of the configuration. This is used for the storage of
NCEP or OGCM data for a specific configuration.
\item ROMSTOOLS\_dir = '../' : "Roms\_tools" directory.
\item RUN\_dir=[ROMSTOOLS\_dir,'Run/'] : Roms\_tools/Run directory. This is where all 
the work is done.
\item ROMS\_files\_dir=[RUN\_dir,'ROMS\_FILES/'] : Roms\_tools/Run/ROMS\_FILES/ directory.
This is where ROMS input NetCDF files are stored.
%
\item grdname=[ROMS\_files\_dir,'roms\_grd.nc'] : Name of the ROMS input NetCDF grid file.
This is where the horizontal grid parameters are stored. In general, we follow 
the style : XXX\_grd.nc.
\item frcname=[ROMS\_files\_dir,'roms\_frc.nc'] : : Name of the ROMS input NetCDF forcing file.
This is where the surface forcing variables (such as wind stress) are stored. In general, we 
follow  the style : XXX\_frc.nc.
\item blkname=[ROMS\_files\_dir,'roms\_blk.nc'] : Name of the ROMS input NetCDF bulk file.
This is where the atmospheric variables used for the bulk parametrization (such as air temperature) 
are stored. In general, we follow  the style : XXX\_blk.nc.
\item clmname=[ROMS\_files\_dir,'roms\_clm.nc'] : Name of the ROMS input NetCDF climatology file.
This is where ROMS prognostic variables (u,v, temp, salt, ubar, vbar, zeta) for lateral boundary 
and interior nudging are stored. This file can be large because variables are stored for all the 
ROMS grid interior points. It is called "a climatology file" because this was the file used in 
the past for the restoring of the ROMS solution towards an in-situ climatology (such as Levitus 
for example). In general, we follow the style : XXX\_clm.nc.
\item ininame=[ROMS\_files\_dir,'roms\_ini.nc'] : Name of the ROMS input NetCDF initial file.
This is where ROMS prognostic variables (u,v, temp, salt, ubar, vbar, zeta) are stored 
for the initial conditions. In general, we follow the style : XXX\_ini.nc.
\item oaname =[ROMS\_files\_dir,'roms\_oa.nc'] : Name of an intermediate file which is not
used by ROMS. This is equivalent to the climatology file, but on a z vertical coordinate.
Firstly, the variables are horizontally interpolated to create a roms\_oa.nc file (a OA file). 
Then, they are vertically interpolated on the ROMS s-coordinate for the climatology
file. In general, we follow the style : XXX\_oa.nc.
\item bryname=[ROMS\_files\_dir,'roms\_bry.nc'] : Name of the ROMS input NetCDF boundary file.
This is an alternative of the climatology file. In this case, variables are only stored for 
the lateral boundaries. In general, we follow the style : XXX\_bry.nc.
\item Zbryname=[ROMS\_files\_dir,'roms\_bry\_Z.nc'] : Intermediate file on a z coordinate
for the boundary file. In general, we follow the style : XXX\_bry\_Z.nc.
\item frc\_prefix=[ROMS\_files\_dir,'roms\_frc'] : First part of the forcing file names in
the case of inter\_annual simulations. In this case, a separate file is created for each month.
For example, a forcing file based on NCEP for January 2000 is : roms\_frc\_NCEP\_Y2000M1.nc
\item blk\_prefix=[ROMS\_files\_dir,'roms\_blk'] : First part of the bulk file names in
the case of inter\_annual simulations. In this case, a separate file is created for each month.
For example, a bulk file based on NCEP for January 2000 is : roms\_blk\_NCEP\_Y2000M1.nc
%
\item Roa=0 : Decorrelation length scale in meters for the objective analysis (300 km
is a reasonable value for the employed datasets). If Roa=0, the "nearest" Matlab extrapolation
method is used instead of an objective analysis. This is much less costly, but the 
results might be at a lower quality.
\item interp\_method = 'cubic' : Horizontal interpolation method used after the objective 
analysis. It can be linear or cubic.
\item makeplot     = 1 : Select to generate images after each step of the preprocessing.
\end{itemize}


\subsection{Building the grid}

The part of the file romstools\_param.m that you should edit is :
\\ \\
\%\%\%\%\%\%\%\%\%\%\%\%\%\%\%\%\%\%\%\%\%\%\\
\%\\
\% 2-Grid parameters\\
\%   used by make\_grid.m (and others..)\\
\%\\
\%\%\%\%\%\%\%\%\%\%\%\%\%\%\%\%\%\%\%\%\%\%\\
\%\\
\% Grid dimensions:\\
\%\\
lonmin =  12.3;   \% Minimum longitude [degree east]\\
lonmax = 20.45;   \% Maximum longitude [degree east]\\
latmin = -35.5;   \% Minimum latitude  [degree north]\\
latmax = -26.5;   \% Maximum latitude  [degree north]\\
\%\\
\% Grid resolution [degree]\\
\%\\
dl = 1/3;\\
\%\\
\% Number of vertical Levels (! should be the same in param.h !)\\
\%\\
N = 32;\\
\%\\
\%  Vertical grid parameters (! should be the same in roms.in !)\\
\%\\
theta\_s = 6.;\\
theta\_b = 0.;\\
hc      =10.;\\
\%\\
\% Minimum depth at the shore [m] (depends on the resolution,\\
\% rule of thumb: dl=1, hmin=300, dl=1/4, hmin=150, ...)\\
\% This affect the filtering since it works on grad(h)/h.\\
\%\\
hmin = 75;\\
\%\\
\% Maximum depth at the shore [m] (to prevent the generation\\
\% of too big walls along the coast)\\
\%\\
hmax\_coast = 500;\\
\%\\
\%  Topography netcdf file name (ETOPO 2 or any other netcdf file\\
\%  in the same format)\\
\%\\
topofile = [ROMSTOOLS\_dir,'Topo/etopo2.nc'];\\
\%\\
\% Slope parameter (r=grad(h)/h) maximum value for topography smoothing\\
\%\\
rtarget = 0.25;\\
\%\\
\% Number of pass of a selective filter to reduce the isolated\\
\% seamounts on the deep ocean.\\
\%\\
n\_filter\_deep\_topo=4;\\
\%\\
\% Number of pass of a single hanning filter at the end of the\\
\% smoothing procedure to ensure that there is no 2DX noise in the \\
\% topography.\\
\%\\
n\_filter\_final=2;\\
\%\\
\%  GSHSS user defined coastline (see m\_map) \\
\%  XXX\_f.mat    Full resolution data\\
\%  XXX\_h.mat    High resolution data\\
\%  XXX\_i.mat    Intermediate resolution data\\
\%  XXX\_l.mat    Low resolution data\\
\%  XXX\_c.mat    Crude resolution data\\
\%\\
coastfileplot = 'coastline\_l.mat';\\
coastfilemask = 'coastline\_l\_mask.mat';\\
\\
Variables description:
\begin{itemize}
\item lonmin =  12.3 : Western limit of the grid in longitude [-360$^\circ$, 360$^\circ$]. 
The grid is rectangular in latitude/longitude.
\item lonmax = 20.45 : Eastern limit [-360$^\circ$, 360$^\circ$]. 
Should be superior to lonmin.
\item latmin = -35.5 : Southern limit of the grid in latitude [-90$^\circ$, 90$^\circ$].
\item latmax = -26.5 : Northern limit [-90$^\circ$, 90$^\circ$].
Should be superior to latmin.
\item l = 1/3 : Grid longitude resolution in degrees. The latitude spacing is deduced to
obtain an isotropic grid using the relation: $d\phi=dl\cos(\phi)$.
\item N = 32 : Number of vertical levels. Warning! N has to be also 
defined in the file : $\sim$/Roms\_tools/Run/param.h before compiling
the model.
\item theta\_s = 6. : Vertical S-coordinate surface stretching parameter. 
When building the climatology and initial ROMS files, we have to define
the vertical grid. Warning! The different vertical grid parameters should 
be identical in this file and in the ROMS input file (i.e. 
$\sim$/Roms\_tools/Run/roms.in).
This is a serious cause of bug.
The effects of theta\_s, theta\_b, hc, and N can be tested 
using the Matlab script : \\
$\sim$/Roms\_tools/Preprocessing\_tools/test\_vgrid.m.
\item theta\_b = 0. : Vertical S-coordinate bottom stretching parameter.
\item hc      = 10. : Vertical S-coordinate $H_c$ parameter. It gives approximately the
transition depth between the horizontal surface levels and the bottom terrain following
levels. It should be inferior to hmin.
\item hmin = 75 : Minimum depth in meters. The model depth is cut a this level 
to prevent, for example, the occurrence of model grid cells without water.
This does not influence the masking routines. At lower resolution, hmin should be 
quite large (for example 150m for dl=1/2). Otherwise, since topography smoothing 
is based on $\frac{\nabla h}{2h}$, the bottom slopes can be totally eroded.
\item hmax\_coast = 500 : Maximum depth under the mask. It prevents selected
isobaths (here 500 m) to go under the mask. If this is the case, 
this could be a source of problems
for western boundary currents (for example).
\item topofile = [ROMSTOOLS\_dir,'Topo/etopo2.nc'] : Default topography file. 
We are using here etopo2 \citep{Smi97}. 
\item rtarget = 0.25 : This variable control the maximum value of the $r$-parameter
that measures the slope of the sigma layers \citep{Bec93}:
$$
r=\frac{\nabla h}{2h}=\frac{h_{+1/2}-h_{-1/2}}{h_{+1/2}+h_{-1/2}}  
$$
To prevent horizontal pressure gradients errors, well known in
terrain-following coordinate models \citep{Han91}, realistic topography
requires some smoothing. Empirical results have shown that reliable
model results are obtained if $r$ does not exceed 0.2.
\item n\_filter\_deep\_topo=4 : Number of pass of a Hanning filter to prevent 
the occurrence of noise and isolated seamounts on deep regions.
\item n\_filter\_final=2 : Number of pass of a Hanning filter at the end of the
smoothing process to be sure that no noise is present in the topography.
\item coastfileplot = 'coastline\_l.mat' : Binary GSHSS coastal file used by m\_map
for graphical pruposes. The letter before ".mat" selects the coastline resolution.
f: Full resolution, h: High resolution, i: Intermediate resolution, l: Low resolution
c: Crude resolution.
\item coastfilemask = 'coastline\_l\_mask.mat' :  Binary file used
for the coastline in the masking toolbox.
\end{itemize}

Save romstools\_param.m and run make\_grid in the Matlab session :
\\ \\ 
$>>$\\
$>>$ make\_grid \\
 \\
You should obtain in the Matlab session:\\
------------------------------------------------------------------------------------------\\
Making the grid: ../Run/ROMS\_FILES/roms\_grd.nc \\
\\
Title: Benguela Test Model \\
\\
Resolution: 1/3 deg \\
\\
Create the grid file... \\
LLm = 23 \\
MMm = 31 \\
\\
Fill the grid file... \\
\\
Compute the metrics... \\
\\
Min dx=30.1583 km - Max dx=33.1863 km \\
Min dy=30.2091 km - Max dy=33.057 km \\
\\
Fill the grid file... \\
\\
Add topography... \\
  ROMS resolution : 31.7 km \\
  Topography data resolution : 3.44 km \\
  Topography resolution halved 4 times \\
   New topography resolution : 54.8 km \\
Processing coastline\_l.mat ... \\
Do you want to use editmask ? y,[n]\\
 Apply a filter on the Deep Ocean to remove the isolated seamounts :\\
   4 pass of a selective filter.\\
 Apply a selective filter on log(h) to reduce grad(h)/h :\\
   13 iterations - rmax = 0.24879\\
 Smooth the topography a last time to prevent 2DX noise:\\
   2 pass of a hanning smoother.\\
 Write it down...\\
 Do a plot...\\
 $>>$\\
 ------------------------------------------------------------------------------------------\\
\\
You should keep the values of LLm and MMm during the process.
They will be necessary for the ROMS parameter file  
$\sim$/Roms\_tools/Run/param.h. In this test case,
LLm0 = 23 and MMm0 = 31. 

During the grid generation process, the question 
"Do you want to use editmask ? y,[n]" is asked. The default answer is n (for no).
If the answer is y (for yes), editmask, the  graphic interface developed 
by A.Y.Shcherbina, will be launched to manually edit the mask 
(Note that, for the moment, editmask is not working with matlab7 and mexnc).
Otherwise the
mask is generated from the unfiltered topography data. A procedure prevents 
the existence of isolated land (or sea) points.

Figure (\ref{fig:grid}) presents the
bottom topography obtained with make\_grid.m for the
Southern Benguela example. Note that at this low
resolution (1/3$^\circ$), the topography has been strongly
smoothed.

\begin{figure}[h!]
\centerline{\psfig{figure=make_grid_benguela.eps,width=6cm}}
\caption{Result of make\_grid.m for the Benguela example}
\label{fig:grid}
\end{figure}


\subsection{Getting the wind and other surface fluxes}

The next step is to create the file containing the different surface
fluxes. The part of the file romstools\_param.m that you should edit is :
\\
\\
\%\\
\%\%\%\%\%\%\%\%\%\%\%\%\%\%\%\%\%\%\%\%\%\%\%\%\%\%\%\\
\% 3-Surface forcing parameters\\
\%   used by make\_forcing.m and by make\_bulk.m\\
\%\\
\%\%\%\%\%\%\%\%\%\%\%\%\%\%\%\%\%\%\%\%\%\%\%\%\%\%\%\\
\% COADS directory (for climatology runs)\\
\%\\
coads\_dir=[ROMSTOOLS\_dir,'COADS05/'];\\
\%\\
\% COADS time (for climatology runs)\\
\%\\
coads\_time=(15:30:345); \% days: middle of each month\\
coads\_cycle=360;        \% repetition of a typical year of 360 days \\ 
\%\\
\%\%\%\%\%\%\%\%\%\%\%\%\%\%\%\%\%\%\%\%\%\%\%\%\%\%\%\\
\%\\
\% 3.1 Surface forcing parameters\\
\%   used by pathfinder\_sst.m\\
\%\\
\%\%\%\%\%\%\%\%\%\%\%\%\%\%\%\%\%\%\%\%\%\%\%\%\%\%\%\\
\%\\
pathfinder\_sst\_name=[ROMSTOOLS\_dir,...\\
                    'SST\_pathfinder/climato\_pathfinder.nc'];\\
\\
Variables description:
\begin{itemize}
\item coads\_dir=[ROMSTOOLS\_dir,'COADS05/'] : Directory where the global atlas of surface marine 
data at 1/2$^\circ$ resolution \citep{Das94} is located.
\item coads\_time=(15:30:345) : Time in days for the monthly climatology. It corresponds to the
middle of each month. ROMS uses this time to interpolate linearly the forcing variables in time.
\item coads\_cycle=360 : Duration on which the forcing variables are cycled. Here, for the sake
of simplicity, we are running the model on a repeating climatological year of 360 days.
\item pathfinder\_sst\_name=[ROMSTOOLS\_dir,SST\_pathfinder/climato\_pathfinder.nc'] : 
Directory of the  monthly climatology of sea surface temperature from Pathfinder satellite 
observations \citep{Cas99}. This can be used has an alternative of \citet{Das94} SST.
\end{itemize}


Save romstools\_param.m and run make\_forcing in the Matlab session :
\\  \\
$>>$\\
$>>$ make\_forcing\\\\
You should obtain :\\
------------------------------------------------------------------------------------------\\
Benguela Test Model\\
\\
 Read in the grid...\\
\\
 Create the forcing file...\\
Getting taux for time index 1\\
Getting tauy for time index 1\\
...\\
Make a few plots...\\
$>>$\\
------------------------------------------------------------------------------------------\\\\
This program can take a relatively long time to process all the forcing variables.
Figure (\ref{fig:forcing}) presents the wind stress vectors and wind stress norm 
obtained from the global atlas of surface marine 
data at 1/2$^\circ$ resolution \citep{Das94} at 4 different periods of the year.
\citet {Das94} sea surface temperature (SST) is used for the restoring term (dQdSST)
in the heat flux calculation. To improve the model solution it is possible to 
use a SST climatology at a finer resolution (9.28 km) \citep{Cas99}. To do 
so, you can run pathfinder\_sst.m in the Matlab session :
\\  \\
$>>$\\
$>>$ pathfinder\_sst\\\\\\\\
You should obtain :\\
------------------------------------------------------------------------------------------\\
 ... Month index: 1 \\
 ... Month index: 2 \\
...\\
$>>$\\
------------------------------------------------------------------------------------------\\\\
For the surface forcing, instead of directly prescribing the fluxes, it is possible 
to use a bulk formula to generate the surface fluxes from atmospheric variables 
during the model run. In this case, ROMS needs to be recompiled with the BULK\_FLUX
cpp key defined. To generate the bulk forcing file, you need to run make\_bulk
in the Matlab session :
\\ \\
$>>$\\
$>>$ make\_bulk\\\\
You should obtain :\\
------------------------------------------------------------------------------------------\\
Benguela Test Model\\
\\
 Read in the grid...\\
\\
 Create the bulk forcing file...\\
Getting sat for time index 1\\
Getting sat for time index 2\\
...\\
Make a few plots...\\
$>>$\\
------------------------------------------------------------------------------------------\\

\begin{figure}[h!]
\centerline{\psfig{figure=make_frcwind_benguela.eps,width=9cm}}
\caption{Wind stress[N.m$^{-2}$] obtained using make\_forcing.m for the Benguela example.}
\label{fig:forcing}
\end{figure}

\subsection{Getting the initial and the lateral boundary conditions}

The last preprocessing step consists in generating the files containing 
the necessary informations for the ROMS initial and lateral open boundaries 
conditions.
This script generates two files : the climatology file (XXX\_clm.nc) which gives 
the lateral boundary conditions, and the initial conditions file (XXX\_ini.nc).
The part which should be edited by the user in the file romstools\_param.m is:
\\ \\
\%\%\%\%\%\%\%\%\%\%\%\%\%\%\%\%\%\%\%\%\%\%\%\%\%\%\%\%\%\%\%\\
\%\\
\% 4-Open boundaries and initial conditions parameters\\
\%   used by make\_clim.m, make\_biol.m, make\_bry.m\\
\%\\
\%\%\%\%\%\%\%\%\%\%\%\%\%\%\%\%\%\%\%\%\%\%\%\%\%\%\%\%\%\%\%\\
\%  Open boundaries switches (! should be consistent with cppdefs.h !)\\
\%\\
obc = [1 1 1 1]; \% open boundaries (1=open , [S E N W])\\
\%\\
\%  Level of reference for geostrophy calculation\\
\%\\
zref = -1000;\\
\%\\
\%  Switches for selecting what to process in make\_clim (1=ON)\\
\%  (and also in make\_OGCM.m and make\_OGCM\_frcst.m)\\
makeini=1;      \%1: process initial data\\
makeclim=1;     \%1: process lateral boundary data\\
makebry=0;      \%1: process boundary data\\
\%\\
makeoa=1;       \%1: process oa data (intermediate file)\\
makeZbry=0;     \%1: process data in Z coordinate\\
\%\\
insitu2pot=1;   \%1: convert in-situ temperature into potential temperature\\
\%\\
\%  Day of initialization for climatology experiments (=0 : 1st January 0h)\\
\%\\
tini=0;\\  
\%\\
\% World Ocean Atlas directory (WOA2001 or WOA2005) \\
\%\\
woa\_dir=[ROMSTOOLS\_dir,'WOA2005/'];\\
\%\\
\% Surface chlorophyll seasonal climatology (WOA2001 or SeaWifs)\\
\%\\
chla\_dir=[ROMSTOOLS\_dir,'SeaWifs/'];\\
\%\\
\%  Set times and cycles for the boundary conditions:\\ 
\%   monthly climatology \\
\%\\
woa\_time=(15:30:345); \% days: middle of each month\\
woa\_cycle=360;        \% repetition of a typical year of 360 days\\  
\%\\

Variables description:
\begin{itemize}
\item obc=[1 1 1 1] : Switches to open (1=open) or close (0=wall) the lateral
boundaries [South East North West]. This is used for the application of mass
enforcement. Be aware, this should be compatible with the open boundary
CPP-switches in the file $\sim$/Roms\_tools/Run/cppdefs.h.
\item zref=-1000 : Depth [meters] of the level of no motion for the geostrophic 
velocities calculation.
\item makeini=1 : Switch to define if the initial file (roms\_ini.nc) is generated. 
Should be 1.
\item makeclim=1 : Switch to define if the climatology
 (lateral boundary conditions) file (roms\_clm.nc) is generated. Should be 1.
\item makeoa=1 : Switch to define if the OA (objective analysis; roms\_oa.nc)
 file is generated. This should be 1. The OA files are intermediate files
 where hydrographic data are stored on a ROMS horizontal grid but on
 a z vertical grid. The transformation into S-coordinate is done later.
 This file is not used by ROMS.
\item makebry=1 : Switch to define if the boundary file (roms\_bry.nc) is generated.
Used only with make\_bry.
\item makeZbry=1  :Switch to define if the boundary intermediate file on a z coordinate 
 (roms\_bry\_Z.nc) is generated. Used only with make\_bry.
\item insitu2pot=1 : Switch defined if it is in-situ temperature that is provided.
In this case, in-situ temperature is converted into potential temperature.
\item tini=0 : Day of initialization in climatology experiments (15 = January 15).
\item woa\_dir=[ROMSTOOLS\_dir,'WOA2005/'] : Directory where the World Ocean
Atlas 2005 climatology \citep{Con02} is located. The World Ocean
Atlas 2001 climatology can also be used.
\item chla\_dir=[ROMSTOOLS\_dir,'SeaWifs/'] : Directory of the surface 
chlorophyll seasonal climatology.
\item woa\_time=(15:30:345) : Time in days for the WOA monthly climatology. 
It corresponds to the middle of each month. ROMS uses this variable to 
interpolate linearly the climatology variables in time.
\item woa\_cycle=360 : Duration on which the climatology variables are cycled. 
Here, for the sake of simplicity, we are running the model on a repeating climatological 
year of 360 days.
\end{itemize}
Save romstools\_param.m and run make\_clim in the Matlab session :
\\ \\
$>>$\\
$>>$ make\_clim \\\\
You should obtain :\\
------------------------------------------------------------------------------------------\\
 Making the clim: ../Run/ROMS\_FILES/roms\_clm.nc \\
 \\
 Title: Benguela Test Model \\
 \\
 Read in the grid...
 \\
 Create the climatology file... \\
 Creating the file : ../Run/ROMS\_FILES/roms\_clm.nc\\
 ...\\ 
 Make a few plots...\\ 
 $>>$\\
------------------------------------------------------------------------------------------\\\\
This program can also take quite a long time to run.
Figure (\ref{fig:clim}) presents 4 different sections
of temperature for the initial condition file for the 
Benguela example. The sections are in the X-direction (East-West), 
the first section is for the Southern part of the domain and the last one 
is for the Northern part of the domain.
\begin{figure}[h!]
\centerline{\psfig{figure=make_clim_benguela.eps,width=7cm}}
\caption{Result of make\_clim.m for the Benguela example}
\label{fig:clim}
\end{figure}

An alternative of using a climatology file is to create a boundary 
file. In this case, only boundary values are stored. The cpp key
FRC\_BRY should be defined and ROMS recompiled. Run make\_bry in 
the Matlab session :
\\ \\
$>>$\\
$>>$ make\_bry \\\\
You should obtain :\\
------------------------------------------------------------------------------------------\\
 Making the file: ../Run/ROMS\_FILES/roms\_bry.nc \\
 \\
 Title: Benguela Test Model \\
 \\
 Read in the grid... \\
... \\
------------------------------------------------------------------------------------------\\

\subsection{Compiling the model}

Once all the netcdf data files are ready (i.e. XXX\_grd.nc,
XXX\_frc.nc, XXX\_ini.nc, and XXX\_clm.nc), we can 
prepare ROMS for compilation. All is done in the 
 $\sim$/Roms\_tools/Run/ directory. 

\subsubsection{param.h}
Edit the file 
 $\sim$/Roms\_tools/Run/param.h.
 The line which needs to be changed is:\\\\
\#  elif defined BENGUELA

      parameter (LLm0=23, MMm0=31, N=32)  ! $<--$ Southern Benguela Test Case\\
\#  else\\
\\
These are the values of the model grid size: LLm0 points in the X 
direction, MMm0 points in the Y direction and N vertical levels.
LLm0 and MMm0 are given by running make\_grid.m, and N is
defined in romstools\_param.m.

\subsubsection{cppdefs.h}
The second file to edit is  $\sim$/Roms\_tools/Run/cppdefs.h.
This file defines the CPP keys that are used by the
the C-preprocessor when compiling ROMS. The C-preprocessor selects the different
parts of the Fortran code which needs to be compiled depending on the defined CPP
options. These options are separated in two parts (the basic option keys and
the advanced options keys) in cppdefs.h.\\

Definitions of the CCP keys in cppdefs.h:
\begin{itemize}
\item BASIN     : Must be defined for running the Basin Example.
\item CANYON\_A  : Must be defined for running the Canyon\_A Example.
\item CANYON\_B  : Must be defined for running the Canyon\_B Example.
\item GRAV\_ADJ  : Must be defined for running the Gravitational Adjustment Example.
\item INNERSHELF   : Must be defined for running the Inner Shelf Example.
\item OVERFLOW  : Must be defined for running the Gravitational/Overflow Example.
\item SEAMOUNT  : Must be defined for running the Seamount Example.
\item SHELFRONT : Must be defined for running the Shelf Front Example.
\item SOLITON   : Must be defined for running the Equatorial Rossby Wave Example.
\item UPWELLING : Must be defined for running the Upwelling Example.
\item VORTEX    : Must be defined for running the Baroclinic Vortex Example  
\item REGIONAL  : Must be defined if running realistic regional simulations.
\\\\\\\\\\
--------------------------\\
       BASIC OPTIONS \\
--------------------------
\\ \\ /*                       Configuration Name */
\item BENGUELA :  Configuration Name, this is used in param.h.
\\ \\ /*                       Parallelization */
\item OPENMP : Activate the Open-MP parallelization protocol.
\item MPI : Activate the MPI parallelization protocol.
\\ \\ /*                       Embedding */
\item AGRIF : Activate the nesting capabilities
\\ \\ /*                       Open Boundary Conditions */
\item TIDES  : Force tidal currents at the lateral boundaries.
\item OBC\_EAST : Open eastern boundary (should be consistent with make\_clim.m).
\item OBC\_WEST : Open western boundary (should be consistent with make\_clim.m).
\item OBC\_NORTH : Open northern boundary (should be consistent with make\_clim.m).
\item OBC\_SOUTH : Open southern boundary (should be consistent with make\_clim.m).
\\ \\ /*                       Embedding conditions */
\item AGRIF\_OBC\_EAST  : Open eastern boundary for the child grids.
\item AGRIF\_OBC\_WEST   : Open western boundary for the child grids.
\item AGRIF\_OBC\_NORTH   : Open northern boundary for the child grids.
\item AGRIF\_OBC\_SOUTH   : Open southern boundary for the child grids.
\\ \\ /*                       Applications */
\item BIOLOGY : Activate the biogeochemical module. 
\item FLOATS : Activate floats.
\item STATIONS : Store model outputs for each time step at different station locations.
\item PASSIVE\_TRACER : Add a passive tracer.
\item SEDIMENT : Activate the sediment module.
\item BBL : Activate the bottom boundary layer module.
\\\\\\\\\\\\\\\\
--------------------------------------------\\
   MORE ADVANCED OPTIONS \\
--------------------------------------------
\\ \\ /*                       Model dynamics */
\item SOLVE3D : Define if solving 3D primitive equations.
\item UV\_COR : Activate Coriolis terms.
\item UV\_ADV : Activate advection terms.
\item SSH\_TIDES : Define for processing sea surface elevation tidal data at the model boundaries.
\item UV\_TIDES :  Define for processing ocean current tidal data at the model boundaries.
\item TIDERAMP  :  Apply a ramping of the tidal current, (in general 2 days) at initialization.
Warning! This should be off when restarting the model.
\\ \\ /*                       Grid configuration */
\item CURVGRID : Activate curvilinear coordinate grid option.
\item SPHERICAL : Activate longitude/latitude grid positioning.
\item MASKING : Activate land masking in the domain.
\\ \\ /*                       Input/Output and Diagnostics */
\item AVERAGES : Define if writing out time-averaged data.
\item AVERAGES\_K : Define if writing out time-averaged vertical mixing.
\item DIAGNOSTICS\_TS  : Define if writing out tendency terms for the tracer equations.
\item DIAGNOSTICS\_UV  : Define if writing out tendency terms for the momentum equations.
\\ \\ /*                       Equation of State */
\item SALINITY : Define if using salinity.
\item NONLIN\_EOS : Activate the nonlinear equation of state.
\item SPLIT\_EOS : Activate the split of the nonlinear equation of state in a
adiabatic part and a compressible part for the reduction of pressure gradient errors
\citep{Shc03a}.
\\ \\ /*                       Surface Forcing */
\item QCORRECTION : Activate net heat flux correction.
\item SFLX\_CORR : Activate freshwater flux correction.
\item DIURNAL\_SRFLUX : Activate diurnal modulation of the short wave radiation flux.
\item BULK\_FLUX : Activate the bulk parametrization.
\item BULK\_EP: Activate the bulk parametrization for salinity fluxes.
\\ \\ /*                       Lateral Forcing */
\item SPONGE : Activate areas of enhanced viscosity/diffusion close to the 
lateral open boundaries.
\item CLIMATOLOGY : Activate processing of climatology data.
\item ZCLIMATOLOGY : Activate processing of  sea surface height climatology.
\item M2CLIMATOLOGY : Activate processing of  barotropic velocities climatology.
\item M3CLIMATOLOGY: Activate processing of  baroclinic velocities climatology.
\item TCLIMATOLOGY : Activate processing of tracer climatology.
\item ZNUDGING : Activate open boundary passive/active term + nudging layer for zeta.
\item M2NUDGING : Activate open boundary passive/active term + nudging layer for ubar and vbar.
\item M3NUDGING : Activate open boundary passive/active term + nudging layer for u and v.
\item TNUDGING : Activate open boundary passive/active term + nudging layer for tracers.
\item ROBUST\_DIAG : Activate strong tracer nudging in the interior for diagnostic simulations.
\item FRC\_BRY : Activate direct boundary forcing (roms\_bry.nc).
\item Z\_FRC\_BRY : Activate boundary forcing for zeta.
\item M2\_FRC\_BRY : Activate boundary forcing for ubar and vbar.
\item M3\_FRC\_BRY : Activate boundary forcing for u and v.
\item T\_FRC\_BRY : Activate boundary forcing for tracers.
\\ \\ /*                       Bottom Forcing */
\item ANA\_BSFLUX : Define if using analytical bottom salinity flux.
\item ANA\_BTFLUX : Define if using analytical bottom temperature flux.
\\ \\ /*                       Point Sources - Rivers */
\item PSOURCE : Define if using point sources (rivers).
\item ANA\_PSOURCE : Define if using analytical vertical profiles for the point sources
(using fluxes defined in roms.in).
\\ \\ /*                       Lateral Mixing */
\item UV\_VIS2 : Activate Laplacian horizontal mixing of momentum.
\item MIX\_GP\_UV : Activate mixing on geopotential (constant Z) surfaces.
\item SMAGORINSKY : Activate Smagorinsky parametrization for horizontal mixing.
\item TS\_DIF2: Activate Laplacian horizontal mixing of tracers.
\item MIX\_GP\_TS : Activate mixing on geopotential (constant Z) surfaces.
\item CLIMAT\_TS\_MIXH : Activate horizontal mixing of T-Tclim instead of T for the tracers.
\\ \\ /*                       Vertical Mixing */
\item BODYFORCE : Define if applying surface and bottom stresses as bodyforces.
\item BVF\_MIXING : Activate a simple mixing scheme based on the Brunt-V\"ais\"al\"a frequency.
\item LMD\_MIXING : Activate Large/McWilliams/Doney mixing (LMD-KPP closure).
\item LMD\_SKPP : Activate surface boundary layer KPP mixing (LMD-KPP closure). 
\item LMD\_BKPP : Activate bottom boundary layer KPP mixing (LMD-KPP closure). 
\item LMD\_RIMIX : Activate shear instability interior mixing (LMD-KPP closure).
\item LMD\_CONVEC : Activate convection interior mixing (LMD-KPP closure).
\item LMD\_DDMIX : Activate double diffusion interior mixing (LMD-KPP closure). 
\item LMD\_NONLOCAL : Activate nonlocal transport (LMD-KPP closure). 
\\ \\ /*                       Open Boundary Conditions */
\item OBC\_M2FLATHER :  Activate Flather open boundary conditions for ubar and vbar.
\item OBC\_M2CHARACT  :  Activate open boundary conditions based on characteristic methods 
for ubar and vbar.
\item OBC\_VOLCONS : Activate mass conservation enforcement at open boundaries.
\item OBC\_M2ORLANSKI : Activate 2D radiation open boundary conditions for ubar and vbar.
\item OBC\_M2SPECIFIED : Activate specified open boundary conditions for ubar and vbar.
\item OBC\_M3ORLANSKI : Activate 2D radiation open boundary conditions for u and v.
\item OBC\_M3CHARACT : Activate open boundary conditions based on characteristic methods 
for u and v.
\item OBC\_M3SPECIFIED : Activate specified open boundary conditions for u and v.
\item OBC\_TORLANSKI :  Activate 2D radiation open boundary conditions for tracers.
\item OBC\_TUPWIND : Activate upwind open boundary conditions for tracers.
\item OBC\_TSPECIFIED : Activate specified open boundary conditions for tracers.
\\ \\ /*                       Embedding conditions */

\item AGRIF\_STORE\_BAROT\_CHILD : Store ubar and vbar during the parent step for the
child boundary conditions (Nesting).
\item AGRIF\_FLUX\_BC : Apply parent/child barotropic boundary conditions as 
fluxes (Nesting).
\item AGRIF\_POLY\_DUAVG : Apply a third order polynomial temporal interpolation 
for parent/child coupling of DU\_avg1 and DU\_avg2 (Nesting).
\item AGRIF\_LOCAL\_VOLCONS : Enforce parent-child mass conservation (Nesting).
\item AGRIF\_OBC\_M2FLATHER :  Activate Flather open boundary conditions for ubar and vbar
for the child model (Nesting).
\item AGRIF\_OBC\_M2ORLANSKI : Activate 2D radiation open boundary conditions for ubar and vbar
for the child model (Nesting).
\item AGRIF\_OBC\_M2SPECIFIED : Activate specified open boundary conditions for ubar and vbar
for the child model (Nesting).
\item AGRIF\_OBC\_M2CHARACT :  Activate open boundary conditions based on characteristic methods 
for ubar and vbar
for the child model (Nesting).
\item AGRIF\_OBC\_M3ORLANSKI : Activate 2D radiation open boundary conditions for u and v
for the child model (Nesting).
\item AGRIF\_OBC\_M3SPECIFIED : Activate specified open boundary conditions for u and v
for the child model (Nesting).
\item AGRIF\_OBC\_M3CHARACT : Activate open boundary conditions based on characteristic methods 
for u and v
for the child model (Nesting).
\item AGRIF\_OBC\_TORLANSKI :  Activate 2D radiation open boundary conditions for tracers
for the child model (Nesting).
\item AGRIF\_OBC\_TUPWIND : Activate upwind open boundary conditions for tracers
for the child model (Nesting).
\item AGRIF\_OBC\_TSPECIFIED  : Activate specified open boundary conditions for tracers
for the child model (Nesting).
\\ \\ /*           Applications */
\\ \\ /*           Biology */
\item BIO\_NChlPZD : Select a 5 components (Nitrate, Chlorophyll, Phytoplankton, Zooplankton,
Detritus) biogeochemical model.
\item BIO\_N2ChlPZD2 : Select a 7 components (Nitrate, Ammonium, Chlorophyll, Phytoplankton, Zooplankton,
Small Detritus, Large Detritus) biogeochemical model. 
\item BIO\_N2P2Z2D2 : Select a 8 components (Nitrate, Ammonium, Small  Phytoplankton, Large Phytoplankton, 
Small Zooplankton, Large Zooplankton,
Small Detritus, Large Detritus) biogeochemical model. 
\item DIAGNOSTICS\_BIO : Define if writing out fluxes between the biological components.
\\ \\ /*           Floats */
\item FLOATS\_GLOBAL\_ATTRIBUTES : Write out global attributes for the floats.
\item IBM  : Add a fish behavior to the floats (Individual Based Model).
\item RANDOM\_WALK : Add a random walk for the floats dispersion.
\item DIEL\_MIGRATION : Add a vertical daily migration to the floats.
\item RANDOM\_VERTICAL : Add a random walk for the floats depending on the vertical mixing.
\\ \\ /*           Stations */
\item ALL\_SIGMA : Write out all vertical levels in the stations file.

\end{itemize}

ROMS can be compiled by running the UNIX tcsh script $\sim$/Roms\_tools/Run/jobcomp.
Jobcomp should be able to recognize your system. It has been tested on 
Linux, IBM, Sun and Compaq systems. On Linux PCs, the default compiler is the GNU g77, 
but it is possible to uncomment specific lines in jobcomp to use g95 or ifort.
The latter is mandatory when using AGRIF and/or OPEN\_MP.
When changing the compiler you should provide a corresponding NetCDF library.
Once the compilation is done, you should obtain a new
executable (roms) in the $\sim$/Roms\_tools/Run directory.
ROMS should be recompiled each time param.h or cppdefs.h are changed.

\subsection{Running the model}

Edit the input parameter file: $\sim$/Roms\_tools/Run/roms.in.
The vertical grid parameters (THETA\_S,   THETA\_B,   HC)
should be identical to the ones in romstools\_param.m.
Otherwise, the other default values should not be changed.
The definition of all the input variables is given at the start of each ROMS
simulation.
To run the model, type in directory $\sim$/Roms\_tools/Run/ : ./roms roms.in.
On the screen, you should check the Cu\_max parameter: if it is greater than
1 you are violating the CFL criterion. In this case, you should reduce the 
time step.
\\
Example of model run:\\
\\
$>$ : ./roms roms.in
\\\\
You should obtain :\\
------------------------------------------------------------------------------------------\\
 Southern Benguela\\
       480  ntimes   Total number of timesteps for 3D equations.\\
   5400.00  dt       Timestep [sec] for 3D equations\\
        60  ndtfast  Number of 2D timesteps within each 3D step.\\
         1  ninfo    Number of timesteps between runtime diagnostics.\\
\\
 6.000E+00  theta\_s  S-coordinate surface control parameter.\\
 0.000E+00  theta\_b  S-coordinate bottom control parameter.\\
 1.000E+01  Tcline   S-coordinate surface/bottom layer width used in\\
                                vertical coordinate stretching, meters.\\
          Grid File:  ROMS\_FILES/roms\_grd.nc\\
  Forcing Data File:  ROMS\_FILES/roms\_frc.nc\\
     Bulk Data File:  ROMS\_FILES/roms\_blk.nc\\
   Climatology File:  ROMS\_FILES/roms\_clm.nc\\
 Initial State File:  ROMS\_FILES/roms\_ini.nc    Record:  1\\
       Restart File:  ROMS\_FILES/roms\_rst.nc    nrst =   480    rec/file:   -1\\
       History File:  ROMS\_FILES/roms\_his.nc  Create new: T  nwrt = 480  rec/file =  0\\
         1  ntsavg      Starting timestep for the accumulation of output\\
                                time-averaged data.\\
        48  navg        Number of timesteps between writing of time-averaged\\
                                data into averages file.\\
      Averages File:  ROMS\_FILES/roms\_avg.nc rec/file =  0\\
\\
 Fields to be saved in history file: (T/F)\\
      T  write zeta  free-surface.\\
      F  write UBAR  2D U-momentum component.\\
      F  write VBAR  2D V-momentum component.\\
      F  write U     3D U-momentum component.\\
      F  write V     3D V-momentum component.\\
      F  write T(1)  Tracer of index 1.\\
      F  write T(2)  Tracer of index 2.\\
\\
      F  write RHO   Density anomaly.\\
      F  write Omega Omega vertical velocity.\\
      F  write W     True vertical velocity.\\
      F  write Akv   Vertical viscosity.\\
      F  write Akt   Vertical diffusivity for temperature.\\
      F  write Aks   Vertical diffusivity for salinity.\\
      F  write Hbl   Depth of KPP-model boundary layer.\\
      F  write Bostr Bottom Stress.\\
\\
 Fields to be saved in averages file: (T/F)\\
      T  write zeta  free-surface.\\
      T  write UBAR  2D U-momentum component.\\
      T  write VBAR  2D V-momentum component.\\
      T  write U     3D U-momentum component.\\
      T  write V     3D V-momentum component.\\
      T  write T(1)  Tracer of index 1.\\
      T  write T(2)  Tracer of index 2.\\
\\
      F  write RHO   Density anomaly\\
      T  write Omega Omega vertical velocity.\\
      T  write W     True vertical velocity.\\
      F  write Akv   Vertical viscosity\\
      T  write Akt   Vertical diffusivity for temperature.\\
      F  write Aks   Vertical diffusivity for salinity.\\
      T  write Hbl   Depth of KPP-model boundary layer\\
      T  write Bostr Bottom Stress.\\
 1025.0000  rho0     Boussinesq approximation mean density, kg/m3.\\
 0.000E+00  visc2    Horizontal Laplacian mixing coefficient [m2/s]\\
                                for momentum.\\
 0.000E+00  tnu2(1)  Horizontal Laplacian mixing coefficient (m2/s)\\
                                for tracer 1.\\
 0.000E+00  tnu2(2)  Horizontal Laplacian mixing coefficient (m2/s)\\
                                for tracer 2.\\
 0.000E+00  rdrg     Linear bottom drag coefficient (m/si).\\
 0.000E+00  rdrg2    Quadratic bottom drag coefficient.\\
 1.000E-02  Zob      Bottom roughness for logarithmic law (m).\\
 1.000E-04  Cdb\_min  Minimum bottom drag coefficient.\\
 1.000E-01  Cdb\_max  Maximum bottom drag coefficient.\\
\\
      1.00  gamma2   Slipperiness parameter: free-slip +1, or no-slip -1.\\
  1.00E+05  x\_sponge Thickness of sponge and/or nudging layer (m)\\
    800.00  v\_sponge Viscosity in sponge layer (m2/s)\\
 1.157E-05  tauT\_in  Nudging coefficients [sec\^-1]\\
 3.215E-08  tauT\_out Nudging coefficients [sec\^-1]\\
 1.157E-06  tauM\_in  Nudging coefficients [sec\^-1]\\
 3.215E-08  tauM\_out Nudging coefficients [sec\^-1]\\
\\
 Activated C-preprocessing Options:\\
\\
          REGIONAL\\
          BENGUELA\\
          OBC\_EAST\\
          OBC\_WEST\\
          OBC\_NORTH\\
          OBC\_SOUTH\\
          SOLVE3D\\
          UV\_COR\\
          UV\_ADV\\
          CURVGRID\\
          SPHERICAL\\
          MASKING\\
          AVERAGES\\
          AVERAGES\_K\\
          SALINITY\\
          NONLIN\_EOS\\
          SPLIT\_EOS\\
          BULK\_FLUX\\
          BULK\_EP\\
          SPONGE\\
          CLIMATOLOGY\\
          ZCLIMATOLOGY\\
          M2CLIMATOLOGY\\
          M3CLIMATOLOGY\\
          TCLIMATOLOGY\\
          ZNUDGING\\
          M2NUDGING\\
          M3NUDGING\\
          TNUDGING\\
          ANA\_BSFLUX\\
          ANA\_BTFLUX\\
          UV\_VIS2\\
          MIX\_GP\_UV\\
          TS\_DIF2\\
          MIX\_GP\_TS\\
          CLIMAT\_TS\_MIXH\\
          LMD\_MIXING\\
          LMD\_SKPP\\
          LMD\_BKPP\\
          LMD\_RIMIX\\
          LMD\_CONVEC\\
          OBC\_M2FLATHER\\
          OBC\_M3ORLANSKI\\
          OBC\_TORLANSKI\\
          M2FILTER\_COSINE\\
\\
Linux 2.6.9-42.0.3.ELsmp x86\_64\\
 NUMBER OF THREADS:  1 BLOCKING:  1 x  1.\\
\\
 Spherical grid detected.\\
\\
 hmin        hmax         grdmin         grdmax         Cu\_min      Cu\_max\\
   75.000000  4803.032721 .301836927E+05 .331215714E+05  0.12176008  0.91533005\\
   volume=9.523986093261087500000E+14   open\_cross=6.104836888312444686890E+09\\
\\
 Vertical S-coordinate System:\\
\\
 level   S-coord     Cs-curve          at\_hmin  over\_slope     at\_hmax\\
\\
    32   0.0000000   0.0000000           0.000       0.000       0.000\\
    31  -0.0312500  -0.0009350          -0.373      -2.584      -4.794\\
    30  -0.0625000  -0.0019030          -0.749      -5.247      -9.746\\
    29  -0.0937500  -0.0029380          -1.128      -8.074     -15.019\\
    28  -0.1250000  -0.0040767          -1.515     -11.152     -20.790\\
    27  -0.1562500  -0.0053591          -1.911     -14.580     -27.249\\
    26  -0.1875000  -0.0068304          -2.319     -18.466     -34.613\\
    25  -0.2187500  -0.0085426          -2.743     -22.938     -43.132\\
    24  -0.2500000  -0.0105560          -3.186     -28.141     -53.095\\
    23  -0.2812500  -0.0129416          -3.654     -34.248     -64.842\\
    22  -0.3125000  -0.0157835          -4.151     -41.463     -78.776\\
    21  -0.3437500  -0.0191819          -4.684     -50.031     -95.377\\
    20  -0.3750000  -0.0232566          -5.262     -60.241    -115.220\\
    19  -0.4062500  -0.0281514          -5.892     -72.443    -138.993\\
    18  -0.4375000  -0.0340388          -6.588     -87.056    -167.524\\
    17  -0.4687500  -0.0411263          -7.361    -104.584    -201.807\\
    16  -0.5000000  -0.0496640          -8.228    -125.635    -243.041\\
    15  -0.5312500  -0.0599527          -9.209    -150.939    -292.668\\
    14  -0.5625000  -0.0723554         -10.328    -181.377    -352.427\\
    13  -0.5937500  -0.0873092         -11.613    -218.013    -424.414\\
    12  -0.6250000  -0.1053416         -13.097    -262.126    -511.156\\
    11  -0.6562500  -0.1270882         -14.823    -315.262    -615.700\\
    10  -0.6875000  -0.1533158         -16.841    -379.282    -741.723\\
     9  -0.7187500  -0.1849493         -19.209    -456.432    -893.656\\
     8  -0.7500000  -0.2231040         -22.002    -549.423   -1076.845\\
     7  -0.7812500  -0.2691252         -25.306    -661.522   -1297.738\\
     6  -0.8125000  -0.3246355         -29.226    -796.670   -1564.114\\
     5  -0.8437500  -0.3915923         -33.891    -959.622   -1885.352\\
     4  -0.8750000  -0.4723564         -39.453   -1156.112   -2272.770\\
     3  -0.9062500  -0.5697755         -46.098   -1393.057   -2740.015\\
     2  -0.9375000  -0.6872846         -54.048   -1678.800   -3303.552\\
     1  -0.9687500  -0.8290268         -63.574   -2023.407   -3983.240\\
     0  -1.0000000  -1.0000000         -75.000   -2439.016   -4803.033\\
\\
 Time splitting: ndtfast = 60    nfast = 89\\
 Maximum grid stiffness ratios:   rx0 =0.2353349875  rx1 =  2.5672736953\\
\\
      GET\_INITIAL -- Processing data for time =   0.000     record =   1\\
\\
      GET\_TCLIMA -- Read climatology of tracer   1 for time =    345.0 \\   
      GET\_TCLIMA -- Read climatology of tracer   1 for time =    15.00 \\   
      GET\_TCLIMA -- Read climatology of tracer   2 for time =    345.0 \\   
      GET\_TCLIMA -- Read climatology of tracer   2 for time =    15.00 \\   
      GET\_UCLIMA -- Read momentum climatology      for time =    345.0 \\   
      GET\_UCLIMA -- Read momentum climatology      for time =    15.00 \\   
      GET\_SSH     - Read SSH climatology           for time =    345.0 \\   
      GET\_SSH     - Read SSH climatology           for time =    15.00 \\   
      GET\_SMFLUX -- Read surface momentum stresses for time =    345.0 \\   
      GET\_SMFLUX -- Read surface momentum stresses for time =    15.00 \\   
      GET\_BULK   -- Read fields for bulk formula   for time =    345.0 \\   
      GET\_BULK   -- Read fields for bulk formula   for time =    15.00 \\   
\\
      DEF\_HIS/AVG - Created new netCDF file 'ROMS\_FILES/roms\_his.nc'.\\
      WRT\_GRID -- wrote grid data into file 'ROMS\_FILES/roms\_his.nc'.\\
      WRT\_HIS -- wrote history fields into time record =   1 /   1\\
\\
 MAIN: started time-steping.\\
\\
 STEP   time[DAYS] KINETIC\_ENRG    POTEN\_ENRG    TOTAL\_ENRG    NET\_VOLUME   trd\\
     0     0.00000 0.000000000E+00 2.1475858E+01 2.1475858E+01 9.5239861E+14  0\\
     1     0.06250 1.306369099E-04 2.1476230E+01 2.1476361E+01 9.5239208E+14  0\\
...\\
------------------------------------------------------------------------------------------\\

\subsection{Long simulations}

In many studies, there is a need for long simulations: to reach the spin-up of 
the solution and/or to obtain statistical equilibriums.
For regional models, 10 years appears to be a reasonable model simulation length.
In this case, to prevent the generation of large output files, the strategy 
is to relaunch the model every simulated month.
This is done by the UNIX csh script: run\_roms.csh .
Warning! the ROMS input file use for long simulations is roms\_inter.in.
It should be edited accordingly.


\begin{enumerate}
\item It gets the grid, the forcing, the initial and the boundary files.
\item It runs the model for 1 month.
\item It stores the output files in a specific form: roms\_avg\_Y4M3.nc (for the ROMS 
averaged output of March of year 4).
\item It replaces the initial file by the restart file (roms\_rst.nc) which as 
been generated at the end of the month.
\item It relaunch the model for next month.
\end{enumerate}

Part to edit in run\_roms.csh:\\
\\
set MODEL=roms \\
set SCRATCHDIR=`pwd`/SCRATCH \\
set INPUTDIR=`pwd` \\
set MSSDIR=`pwd`/ROMS\_FILES \\
set MSSOUT=`pwd`/ROMS\_FILES \\
set CODFILE=roms \\
set AGRIF\_FILE=AGRIF\_FixedGrids.in \\
\# \\
\# Model time step [seconds] \\
\# \\
set DT=5400 \\
\# \\
\# Number of days per month \\
\# \\
set NDAYS = 30 \\
\# \\
\# number total of grid levels \\
\# \\
set NLEVEL=1 \\
\# \\
\#  Time Schedule  -  TIME\_SCHED=0 --$>$ yearly files \\
\#                    TIME\_SCHED=1 --$>$ monthly files \\
\# \\
set TIME\_SCHED=1 \\
\# \\
set NY\_START=1 \\
set NY\_END=10 \\
set NM\_START=1 \\
set NM\_END=12 \\
\\

Variables definitions:
\begin{itemize}
\item MODEL=roms : Name used for the input files. For example roms\_grd.nc.
\item SCRATCHDIR=`pwd`/SCRATCH : Scratch directory where the model is run
\item INPUTDIR=`pwd` : Input directory where the roms\_inter.in input file
is.
\item MSSDIR=`pwd`/ROMS\_FILES : Directory where the roms input NetCDF files
(roms\_grd.nc, roms\_frc.nc, ...) are stored.
\item MSSOUT=`pwd`/ROMS\_FILES : Directory where the roms output NetCDF files
(roms\_his.nc, roms\_avg.nc, ...) are stored.
\item CODFILE=roms : ROMS executable.
\item AGRIF\_FILE=AGRIF\_FixedGrids.in : AGRIF input file which defines the 
position of child grids when using embedding.
\item DT=5400 : Model time step in seconds.
\item NDAYS = 30 : Number of days in 1 month.
\item NLEVEL=1 : Total number of model grids (no AGRIF: NLEVEL=1).
\item NY\_START=1 : Starting year.
\item NY\_END=10 : Ending Year.
\item NM\_START=1 : Starting month.
\item NM\_END=12 : Ending month.
\end{itemize}

To run a ROMS long simulation in batch mode on a Linux workstation:\\	

$>$ : nohup ./run\_roms.csh $>$ exp1.out \&\\\\
To check the execution of your model, type in the directory
$\sim$/Roms\_Tools/Run :\\ $>$: more exp1.out
\subsection{Getting the results}

\subsubsection{roms\_gui}

Once the model has run, or during the simulation, it is possible
to visualize the model outputs using a Matlab graphic user interface : 
roms\_gui. Launch roms\_gui in the Matlab session 
(in the  $\sim$/Roms\_tools/Run/ directory):
\\ \\
$>>$ \\
$>>$ roms\_gui
\\\\
A window pops up, asking for a ROMS history NetCDF file (Figure \ref{fig:open}).
You should select roms\_his.nc (history file) or roms\_avg.nc (average file) and
click "open".
\begin{figure}[!ht]
\centerline{\psfig{figure=roms_gui_select.eps,width=5cm}}
\caption{Entrance window of roms\_gui}
\label{fig:open}
\end{figure}
\\
\begin{figure}[!ht]
\centerline{\psfig{figure=roms_gui.eps,width=9cm}}
\caption{roms\_gui}
\label{fig:romsgui}
\end{figure}

The main window appears, variables can be selected to obtain an image such as
Figure (\ref{fig:romsgui}). On the left side, the upper box  gives the available 
ROMS variable names and the lower box presents the variables derived from the
ROMS model outputs :

\begin{itemize}
\item Ke : Horizontal slice of kinetic energy: $0.5(u^2+v^2)$.
\item Rho : Horizontal slice of density using the non-linear equation of state 
for seawater of \citet{Jac95}.
\item Pot\_Rho : Horizontal slice of the potential density.
\item Bvf : Horizontal slice of the Brunt-V\"ais\"ala frequency: 
$N^2=-\frac{g}{\rho}\frac{\partial \rho}{\partial z}$ 
\item Vort : Horizontal slice of vorticity: $\frac{\partial v}{\partial x}-
\frac{\partial u}{\partial y}$.
\item Pot\_vort : Horizontal slice of the vertical component of Ertel's potential vorticity:
$\frac{\partial \lambda}{\partial z} \left [ f + 
\left (\frac{\partial v}{\partial x}-\frac{\partial u}{\partial y}\right ) \right ]$.
In our case, $\lambda=\rho$.
\item Psi : Horizontal slice of stream function: 
$\nabla^2 \psi=\frac{\partial v}{\partial x}-\frac{\partial u}{\partial y}$.
This routine might be costly since it inverses the Laplacian of the vorticity
(using a successive over relaxation solver).
\item Speed : Horizontal slice of the ocean currents velocity : $\sqrt{u^2+v^2}$.
\item Transport : Horizontal slice of the transport stream function : 
$\nabla^2 S_{vd}=\frac{\partial \bar{v}}{\partial x}-\frac{\partial \bar{u}}{\partial y}$.
\item Okubo : Horizontal slice of the Okubo-Weiss parameter : 
$\Lambda^2=
\left ( \frac{\partial u}{\partial x}-\frac{\partial v}{\partial y} \right )^2+
\left ( \frac{\partial v}{\partial x}+\frac{\partial u}{\partial y} \right )^2-
\left ( \frac{\partial v}{\partial x}-\frac{\partial u}{\partial y} \right )^2$.
\item Chla : Compute a chlorophyll-a from Large and Small phytoplankton concentrations.
\item z\_SST\_1C : Depth of 1$^\circ$C below SST.
\item z\_rho\_1.25 : Depth of 1.25 kg.m$^{-3}$ below surface density.
\item z\_max\_bvf : Depth of the maximum of the  Brunt-V\"ais\"ala frequency.
\item z\_max\_dtdz : Depth of the maximum  vertical temperature gradient.
\item z\_20C : Depth of the 20$^\circ$C isotherm.
\item z\_15C : Depth of the 15$^\circ$C isotherm.
\item z\_sig27 : Depth of the 1027 kg.m$^{-3}$ density layer.
\item r\_factor : $r=\frac{\nabla h}{2h}=\frac{h_{+1/2}-h_{-1/2}}{h_{+1/2}+h_{-1/2}}$
\end{itemize}

It is possible to add arrows for the horizontal currents by increasing the "Current vectors 
spatial step". It is also possible to obtain vertical sections, time series, vertical profiles
and Hovm\"uller diagrams by clicking on the corresponding targets in roms\_gui.

\subsubsection{Diagnostics}

To analyze the long simulations,
a few scripts have been added in the directory: \\
$\sim$/Roms\_tools/Diagnostic\_tools:
\begin{itemize}
\item roms\_diags.m : Get volume and surface averaged quantities from a ROMS simulation.
\item plot\_diags.m :  Plot the averaged quantities computed by roms\_diags.m.
\item get\_Mmean.m  : Get the monthly mean climatology.
\item get\_Smean.m  : Get the seasonal and annual mean climatology from the outputs of
get\_Mmean.m.
\item get\_Meddy.m  : Get the monthly variance climatology (if the variable nonseannal = 1, 
the non-seasonal variance is computed; i.e., the seasonal variation are 
filtered). It needs that get\_Mmean.m and get\_Smean.m are run before.
\item get\_Seddy.m  : Get the seasonal and annual RMS from the results of 
get\_Meddy.m.
\item roms\_anim.m  : Create an animation from the monthly  history or average files.
\end{itemize}
Run these scripts in a Matlab session.
The obtained mean or eddy files can be visualized with roms\_gui.\\

If you need to create and play ".fli" animations, you should install ppm2fli
and xanim on your system. If you have a Linux PC, you can follow these steps:
\begin{enumerate}
\item log in as root
\item go to the directory where the file is saved.
\item type : rpm -Uvh  ppm2fli-2.1-1.i386.rpm
\item type : rpm -Uvh  xanim-2.80.1-12.i386.rpm
\item log out
\end{enumerate}
If you are not using a Linux PC, you should ask your 
system administrator to install these programs.\\
